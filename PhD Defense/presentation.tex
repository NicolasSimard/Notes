\documentclass{beamer}
\usetheme{Singapore}

% --------------------------------------------------------------------
% Voici certains packages souvent utilisés.
\usepackage{graphicx}		% Importation du package permettant d'inclure d'images dans le document.
\usepackage{amsmath, amsfonts}	% Pour écrire selon les standards de l'AMS.
\usepackage{epstopdf}		% Package permettant l'inclusion d'images eps, pour la compilation en pdf.
\usepackage{epsfig}
\usepackage{hyperref}
%\usepackage{subfig}

\usepackage{multirow}
\usepackage{makecell}
\usepackage{xcolor, colortbl}
\definecolor{Green}{RGB}{156, 255, 173}
\definecolor{Red}{RGB}{255, 173, 156}

% --------------------------------------------------------------------
% --------------------------------------------------------------------
% Commande rapide
\newcommand{\C}{\mathbb{C}}
\newcommand{\F}{\mathbb{F}}
\newcommand{\R}{\mathbb{R}}
\newcommand{\Q}{\mathbb{Q}}
\newcommand{\N}{\mathbb{N}}
\newcommand{\Z}{\mathbb{Z}}
\newcommand{\cH}{\mathcal{H}}
\newcommand{\cW}{\mathcal{W}}
\newcommand{\cO}{\mathcal{O}}
\newcommand{\cA}{\mathcal{A}}

\newcommand{\abcdmat}{\begin{pmatrix}
a & b \\ 
c & d
\end{pmatrix}}
\newcommand{\fa}{\mathfrak{a}}
\newcommand{\fb}{\mathfrak{b}}
\newcommand{\fc}{\mathfrak{c}}
\newcommand{\fp}{\mathfrak{p}}
\newcommand{\td}{\text{d}}
\newcommand{\tpsi}{\theta_\psi}
\newcommand{\ClK}{\text{Cl}_K}

\DeclareMathOperator{\Sym}{Sym}
\DeclareMathOperator{\Ind}{Ind}
\DeclareMathOperator{\Gal}{Gal}
\DeclareMathOperator{\Frob}{Frob}
\DeclareMathOperator{\Hom}{Hom}
\DeclareMathOperator{\res}{res}

\newtheorem{thm}{Theorem}
\newtheorem{defn}[thm]{Definition}
\newtheorem{prop}[thm]{Proposition}
\newtheorem{coro}[thm]{Corollary}
\newtheorem{lem}[thm]{Lemma}
\newtheorem{conj}[thm]{Conjecture}
\newtheorem{rem}[thm]{Remark}
\newtheorem{ex}[thm]{Example}
\newtheorem{qu}{Question}

\DeclareMathOperator{\reg}{reg}

\title{Petersson Inner Product of Theta Series}
\subtitle{PhD Defense}
\author{Nicolas \textsc{Simard}}
\institute{McGill University}
\date{April 12th, 2018}

\begin{document}

\frame{
	\titlepage
}

\section{Introduction}
\frame{
	\frametitle{$L$-functions at $s=1$}
	It is a well-known (but fascinating) fact that many $L$-functions contain arithmetic informations in their value at $s=1$:
	\begin{enumerate}
		\item $\zeta(s)$ at $s=1$: Infinitely many primes
		\item $L(\chi, s)$ at $s=1$: Infinitely many primes in arithmetic progressions
		\item $\zeta_F(s)$ at $s=1$: Class number formula
	\end{enumerate}
	\only<2->{
	\begin{conj}[Stark (Idea)]
		In general, $L$-functions of Artin representations have a (relatively) explicit expression involving arithmetic invariants of the number fields involved. 
	\end{conj}
	}
}

\frame{
	\frametitle{An observation of Stark}
	Let $K=\Q(\sqrt{-23})$ and let $H$ be its Hilbert class field.
	Let
	\[\psi:\Gal(H/K)\rightarrow \C^\times = \text{GL}_1(\C)\]
	be a non-trivial one-dimensional Artin representation and let
	\[\rho =\Ind_{K}^{\Q}\psi:\Gal(H/\Q)\rightarrow\text{GL}_2(\C)\]
	be the induced representation. Then one can consider the associated Artin $L$-function
	\[L(\psi, s) = L(\rho, s).\]
}

\frame{
	\frametitle{An observation of Stark}
	On the one hand, in accordance with his conjecture (which was known in this case), Stark shows that
	\[L(\rho, 1) = \frac{2\pi}{\sqrt{23}}\log\varepsilon,\]
	where $\varepsilon$ is the real root of
	\[x^3-x-1.\]
	Note that $\varepsilon$ generates $H$ over $K$.
}

\frame{
	\frametitle{An observation of Stark}
	On the other hand, by the Deligne-Serre theorem, one has
	\[L(\rho, s) = L(\theta_\psi,s),\]
	where
	\[\tpsi(q) = \eta(q)\eta(23q) = q\prod_{n\geq 1}(1-q^n)(1-q^{23n})\in M_1(\Gamma_0(23),\chi_{-23}).\]
	Then Stark proves that
	\[L(\rho, 1) = \frac{2\pi}{3\sqrt{23}}\langle\tpsi,\tpsi\rangle.\]
}

\frame{
	\frametitle{The main motivation}
	It follows that
	\[\langle\tpsi,\tpsi\rangle = 3\log\varepsilon.\]
}

\frame{
	\frametitle{Structure of the presentation}
	\tableofcontents
}

\AtBeginSection[]
{
  \begin{frame}
    \frametitle{Where we are in the presentation}
    \tableofcontents[currentsection]
  \end{frame}
}

\section[Formulas]{Petersson inner product of theta series}
\frame{
	\frametitle{Notation}
	Throughout this presentation, let
	\begin{itemize}
		\item $K$ be an imaginary quadratic field of discriminant $D$ with Hilbert class field $H$,
		\item $h_K, w_K$ and $\ClK$ be the class number, root number and class group of $K$ (respectively)
		\item $\psi$ be a Hecke character of infinity type $(2\ell, 0)$ for some $\ell\geq 0$, i.e. a homomorphism
		\[\psi:I_K\longrightarrow\C^\times\]
		such that $\psi((\alpha)) = \alpha^{2\ell}$ for all $\alpha\in K^\times$
		\item and $\fa, \fb$ and $\fc$ be fractional ideals of $K$.
	\end{itemize}
}

\frame{
	\frametitle{Theta series attached to $K$}
	Consider
	\[
	\left. \begin{tabular}{rl}
		$\theta_\psi(q)$ & = $\sum_{\fa\in\cO_K}\psi(\fa)q^{N(\fa)}$\\
		&\\
		$\theta_\fa(q)$  & = $\sum_{x\in\fa} x^{2\ell}q^{N(x)/N(\fa)}$
	\end{tabular}\right \} \in M_{2\ell+1}(\Gamma_0(|D|), \chi_D).
	\]
	Then
	\begin{center}
	\renewcommand{\arraystretch}{1.5}
	\begin{tabular}{|>{\centering\arraybackslash\hspace{0pt}}p{0.1\linewidth}|>{\centering\arraybackslash\hspace{0pt}}p{0.4\linewidth}|>{\centering\arraybackslash\hspace{0pt}}p{0.4\linewidth}|}
    	\cline{2-3}
    	\multicolumn{1}{c|}{}& $\theta_\psi$ & $\theta_{\fa,\ell}$\\
   		\hline
    	$\ell>0$&\cellcolor{Green}Newform&\cellcolor{Green}Cusp form\\
    	\hline
    	\multirow{3}{*}{$\ell=0$}&\cellcolor{Green}$\psi^2\neq1$: Newform&\cellcolor{Red}\\
    	\cline{2-2}
    	&\cellcolor{Red}$\psi^2=1$: (genus) Eisenstein series&\multirow{-1}{*}{\cellcolor{Red}Not a cusp form}\\
    	\hline
	\end{tabular}
	\end{center}
}

\frame{
	\frametitle{Some examples to keep in mind}
	\begin{center}
	\renewcommand{\arraystretch}{1.5}
	\begin{tabular}{|>{\centering\arraybackslash\hspace{0pt}}p{0.1\linewidth}|>{\centering\arraybackslash\hspace{0pt}}p{0.4\linewidth}|>{\centering\arraybackslash\hspace{0pt}}p{0.4\linewidth}|}
    	\cline{2-3}
    	\multicolumn{1}{c|}{}& $\theta_\psi$ & $\theta_{\fa,\ell}$\\
   		\hline
    	$\ell>0$&\cellcolor{Green}&\cellcolor{Green}\\
    	\hline
    	\multirow{3}{*}{$\ell=0$}&\cellcolor{Green}$q\prod_{n\geq 1}(1-q^n)(1-q^{23n})$&\cellcolor{Red}\\
    	\cline{2-2}
    	&\cellcolor{Red}&\multirow{-2}{*}{\cellcolor{Red}$\theta_{\Z[i]}(q)=\sum\limits_{x,y\in\Z}q^{x^2+y^2}$}\\
    	\hline
	\end{tabular}
	Recall that
	\[q\prod_{n\geq 1}(1-q^n)(1-q^{23n})\]
	is the modular form in Stark's example.
	\end{center}
}

\frame{
	\frametitle{Formulas for the Petersson inner product of those theta series}
	Recall that the Petersson inner product of any cusp forms $f, g\in S_k(\Gamma_0(N), \chi)$ is defined as
	\[\langle f, g\rangle = \iint_{\Gamma_0(N)\setminus \cH}f(\tau)\bar{g}(\tau)\Im(\tau)^{k}\td\mu(\tau).\]
	\only<2->{With minor effort, this formula can be used to compute the Petersson inner product numerically:
	\[\langle f, g\rangle = \sum_{\gamma\in\Gamma_0(N)\setminus\cH}\int_{-\frac{1}{2}}^{\frac{1}{2}}\int_{\sqrt{1-x^2}}^\infty f(\tau)\bar{g}(\tau)y^{k-2}\td y\td x.\]
	But this is very (very) slow and behaves badly as the level grows.}	
}

\frame{
	\frametitle{The quest for more efficient and useful formulas}
	Let $\psi$ be such that $\theta_\psi$ is a cusp form. Then
	\begin{enumerate}[<+->]
		\item Apply Rankin-Selberg:
		\[\langle \theta_\psi,\theta_\psi\rangle = \textcolor{gray}{\left (\dfrac{\pi}{2}\dfrac{\phi(|D|)}{D^2}\dfrac{(4\pi)^{2\ell+1}}{\Gamma(2\ell+1)}\right )^{-1}L(\chi_D,1)}\res\limits_{s=2\ell+1}L(\Sym^2 \theta_\psi,s)\]
		\item Isolate the residue of $L(\Sym^2 \theta_\psi,s)$:
		\[\res_{s=2\ell+1}L(\Sym2 \theta_\psi,1,s)=\textcolor{gray}{\prod_{p|D}(1-p^{-1})}L(\psi^2,2\ell+1)\]
		\item When $\ell>0$, express $L(\psi^2,2\ell+1)$ in terms of (derivatives of nearly holomorphic) Eisenstein series:
		\[L(\psi^2,2\ell+1) = \textcolor{gray}{\frac{4(2\pi)^{2\ell+1}\sqrt{|D|}^{2\ell-1}}{w_K\Gamma(2\ell+1)}}\sum_{j=1}^{h_K}\psi^{-2}(\fa_j)N(\fa_j)^{4\ell}\delta^{2\ell-1} E_2(\bar{\fa}_j)\]
	\end{enumerate}
}

\frame{
	\frametitle{The most useful formulas for $p$-adic interpolation}
	\begin{center}
	\renewcommand{\arraystretch}{1.5}
	\begin{tabular}{|>{\centering\arraybackslash\hspace{0pt}}p{0.1\linewidth}|>{\centering\arraybackslash\hspace{0pt}}p{0.4\linewidth}|>{\centering\arraybackslash\hspace{0pt}}p{0.4\linewidth}|}
    	\cline{2-3}
    	\multicolumn{1}{c|}{}& $\langle\theta_\psi,\theta_\psi\rangle$ & $\langle\theta_{\fa,\ell},\theta_{\fb, \ell}\rangle$\\
   		\hline
    	\newline$\ell>0$
    	&\cellcolor{Green}$C_1\sum\limits_{\cA\in\ClK}\psi^2(\cA)\delta^{2\ell-1} E_2(\cA)$
    	&\cellcolor{Green}\newline\only<2->{$C_2\sum\limits_{\fa\bar{\fb}\fc^2=\lambda_\fc\cO_K}\lambda_\fc^{2\ell}\delta^{2\ell-1}E_2(\fc)$}
    	\\
    	\hline
    	\multirow{2}{*}{$\ell=0$}&\cellcolor{Green}\only<3->{$C_3\sum\limits_{\cA\in\ClK}\psi^2(\cA)\log \Phi(\cA)$}&\cellcolor{Red}\\
    	\cline{2-2}
    	&\cellcolor{Red}$\psi^2=1$: not applicable &\multirow{-2}{*}{\cellcolor{Red}not applicable}\\
    	\hline
	\end{tabular}
	\end{center}
	\only<2>{
		Using the relation
		\[\theta_{\fa,\ell}=\frac{w_K}{h_K}\sum_{\psi}\psi(\fa)\tpsi\]
		and the orthogonality of the newforms $\tpsi$.}
	\only<3>{
		Here
		\[\Phi(\cA) = N(\cA)^6|\Delta(\cA)|,\]
		where
		\[\Delta(q) = q\prod_{n=1}^\infty(1-q^n)^{24}.\]}
}

\section{Algorithms}
\frame{
	\frametitle{Bridging the gap between the "explicit" formulas and the algorithms}
	Here are some of the things one needs to do before implementing those formulas:
	\begin{itemize}
		\item Complete the $L$-functions $L(\Sym^2\tpsi, s)$ and $L(\psi, s)$ and find all the information about their functional equation,
		\item Find a way to compute with Hecke characters,
		\item Find an \emph{efficient} way to compute
		\[\delta^nE_2(\fa),\]
		\item Choose the computer algebra system that allows you to do all this!
	\end{itemize}
}

\frame{
	\frametitle{The most efficient formula for computations}
	Experimentally, one finds that the most efficient way to compute the Petersson inner product of theta series is to compute the $q$-expansion of $\delta^nE_2$ by hand:
	\begin{align*}
	\begin{split}
		\delta^nE_2(\tau) = &(-1)^n\left (\frac{1}{8\pi\Im(\tau)} - \frac{n + 1}{24}\right )\frac{n!}{(4\pi\Im(\tau))^n}\\
							&\qquad{} +\sum_{m\geq 1}\sigma(m)\left (\sum_{r=0}^n(-1)^{n-r}\binom{n}{r}\frac{(r+2)_{n-r}}{(4\pi\Im(\tau))^{n-r}}m^r\right )q^m.
	\end{split}
	\end{align*}
}

\frame{
	\frametitle{The resulting algorithm}
	This leads to the following
	\begin{theorem}[S.]
		There exists a software package to compute the Petersson inner product of the theta series defined above with the following properties:
		\begin{itemize}
			\item It is fast (relative to the definition),
			\item It supports arbitrary precision (no coefficients stored, no database involved),
			\item User friendly (easy to download, help functions, well commented source code),
		\end{itemize}
	\end{theorem}
	\begin{proof}
		See the calculations at the end of the thesis!
	\end{proof}
}


\section[Stark's observation]{Generalizations of Stark's observation}

\section[$p$-adic]{$p$-adic interpolation}

\section[Experiments]{Experimentation and observations}

\section{Conclusion}


\end{document}
