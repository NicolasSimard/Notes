\documentclass{beamer}
\usetheme{Singapore}

% --------------------------------------------------------------------
% Packages pour la couleur
\usepackage{color}

% --------------------------------------------------------------------
% Packages pour les accents français.
\usepackage[latin1]{inputenc}	% L'option "latin1" désigne l'encodage ISO-8859-1, typique sous Linux.
\usepackage[T1]{fontenc}	% Pour les accents
\usepackage[frenchb]{babel}

% --------------------------------------------------------------------
% Voici certains packages souvent utilisés.
\usepackage{graphicx}		% Importation du package permettant d'inclure d'images dans le document.
\usepackage{amsmath, amsfonts}	% Pour écrire selon les standards de l'AMS.
\usepackage{epstopdf}		% Package permettant l'inclusion d'images eps, pour la compilation en pdf.
\usepackage{epsfig}
\usepackage{hyperref}
%\usepackage{subfig}
\usepackage{multirow}

% --------------------------------------------------------------------
% Le package "palatino" charge la police Palatino en mode texte et le package "euler" charge la police Euler en mode mathématique. Pour retrouver les polices par défaut, effacez les deux lignes de commandes qui suivent.
\usepackage{palatino}
\usepackage{euler}

% --------------------------------------------------------------------
% --------------------------------------------------------------------
% Commande rapide
\newcommand{\C}{\mathbb{C}}
\newcommand{\F}{\mathbb{F}}
\newcommand{\R}{\mathbb{R}}
\newcommand{\Q}{\mathbb{Q}}
\newcommand{\N}{\mathbb{N}}
\newcommand{\Z}{\mathbb{Z}}
\renewcommand{\H}{\mathcal{H}}
\renewcommand{\O}{\mathcal{O}}

\newcommand{\abcdmat}{\begin{pmatrix}
a & b \\ 
c & d
\end{pmatrix}}
\newcommand{\ida}{\mathfrak{a}}
\newcommand{\idb}{\mathfrak{b}}
\newcommand{\idc}{\mathfrak{c}}
\newcommand{\tha}{\theta_\mathfrak{a}}
\newcommand{\thb}{\theta_\mathfrak{b}}
\newcommand{\tpsi}{\theta_\psi}
\newcommand{\ClK}{\text{Cl}_K}

\newtheorem{thm}{Theorem}
\newtheorem{defn}[thm]{Definition}
\newtheorem{prop}[thm]{Proposition}
\newtheorem{coro}[thm]{Corollary}
\newtheorem{lem}[thm]{Lemma}
\newtheorem{conj}[thm]{Conjecture}
\newtheorem{rem}[thm]{Remark}
\newtheorem{ex}[thm]{Example}

%Groupe modulo
\newcommand*{\modulo}[2]
{\raisebox{.6ex}{ \newline \ensuremath{#1}}\!/\!\raisebox{-.6ex}{\ensuremath{#2}}}

% --------------------------------------------------------------------
% --------------------------------------------------------------------
% --------------------------------------------------------------------

\title{Petersson Inner Product of Binary Theta Series}
\subtitle{A computational approach}
\author{Nicolas \textsc{Simard}}
\institute{McGill University}
\date{September 17th, 2016}


\begin{document}

\begin{frame}
\titlepage
\end{frame}

\AtBeginSection[]
{
  \begin{frame}
    \frametitle{Table of Contents}
    \tableofcontents[currentsection]
  \end{frame}
}

\section{Background and setup}
\subsection{Modular forms}

\begin{frame}
\frametitle{Mobius transformations}
Let $\H$ be the Poincarre upper-half plane. Recall that $\text{GL}_2(\R)_+$ acts on $\H$ via Mobius transformations:
\[\abcdmat z=\frac{az+b}{cz+d}.\]
\begin{defn}
	Let $N\geq1$ and define the Hecke subgroup of level $N$ as
	\[\Gamma_0(N)=\left\lbrace\abcdmat \in\text{SL}_2(\Z)| c\equiv 0\pmod N\right\rbrace.\]
\end{defn}
	
\end{frame}

\begin{frame}
\frametitle{Level $N$ modular forms with characters}

\begin{defn}
	Let $N\geq1$ and $k\geq0$ be integers and let $\chi$ be a Dirichlet character mod $N$. A modular form of weight $k$, level $N$ and character $\chi$ is a holomorphic function
	\[f:\H\longrightarrow\C\]
	such that
	\[f\left(\gamma z\right)=\chi(d)(cz+d)^{-k}f(z)\]
	for all $z\in\H$ and all $\gamma\in\Gamma_0(N)$, which satisfies certain growth conditions at the cusps.	The $\C$-vector-space of such modular forms is denoted
	\[M_k(\Gamma_0(N),\chi).\]
\end{defn}
	
\end{frame}	
	
\frame{
	\frametitle{$q$-expansion of modular forms}
	Every modular form $f$ has a Taylor (or Fourrier) expansion at infinity, called its $q$-expansion:
	\[f(z)=\sum_{n=0}^\infty a_nq^n,\]
	where $q=exp(2\pi iz)$. If
	\[a_0(f)=0,\]
	(at all cusps) $f$ is called a \emph{cusp form}.
}

\frame{
	\frametitle{Example: weight $k$ Eisenstein series}
	Let $k\geq4$ be an even integer and define
	\[G_k(z)=\sum_{m,n}\frac{1}{(mz+n)^k}\in M_k(\Gamma_0(1),1).\]
	After renormalisation, the $q$-expansion of $G_k$ is
	\[E_k(z) = -\frac{B_k}{2k}+\sum_{n=1}^\infty\sigma_{k-1}(n)q^n.\]
}

\frame{
	\frametitle{Important non-example: weight $2$ Eisenstein series}
	In level $1$, there are no modular forms of weight $2$. However, one can still define the weight $2$ Eisenstein series as
	\[E_2(2)=\frac{1}{8\pi\Im(z)}-\frac{1}{24}+\sum_{n=1}^\infty\sigma(n)q^n.\]
	It is an example of an \emph{almost holomorphic} modular form of level $1$ and weight $2$.
}

\subsection{Spaces of modular forms}
\frame{
	\frametitle{Spaces of modular forms}
	\begin{itemize}
		\item<1-> $M_k(\Gamma_0(N),\chi)$ is finite dimensional.
		\item<2-> For every integer $n\geq1$, one can define a \emph{Hecke operator} $T_n$ (depending on $k$, $N$ and $\chi$) which acts on $M_k(\Gamma_0(N),\chi)$.
		\item<3-> There exists a basis of common eigenvectors for all Hecke operators $T_n$ with $(n,N)=1$.
	\end{itemize}
}

\frame{
	\frametitle{Petersson inner product}
	Let $f,g\in S_k(\Gamma_0(N),\chi)$ be two cusp forms. The Petersson inner product of $f$ and $g$ is defined as
	\[\langle f,g\rangle =\frac{1}{\text{Vol}(\Gamma_0(N)\setminus\H)}\int_{\Gamma_0(N)\setminus\H}f(x+iy)\overline{g(x+iy)}y^k\text{d}\mu,\]
	where
	\[\text{d}\mu=\frac{\text{d}x\text{d}y}{y^2}\]
	is the $\text{SL}_2(\R)$-invariant measure on $\H$.
	Note that the integral does not converge if neither $f$ nor $g$ is a cusp form.
}

\subsection{Newforms}
\frame{
	\frametitle{Newforms}
	The space $S_k(\Gamma_0(N),\chi)$ splits naturally as
	\[S_k(\Gamma_0(N),\chi)=S_k(\Gamma_0(N),\chi)^{\text{new}}\oplus S_k(\Gamma_0(N),\chi)^\text{old}.\]
	\begin{theorem}
		The space $S_k(\Gamma_0(N),\chi)^{\text{new}}$ has an \emph{orthogonal} basis of eigenvectors for \emph{all} Hecke operators. Elements of this basis are called newforms (after suitable normalization).
	\end{theorem}
}

\frame{
	\frametitle{Summary}
	\begin{enumerate}
		\item<1-> The space $S_k(\Gamma_0(N),\chi)$ is a finite dimensional Hermitian inner product space, equipped with an action of Hecke operators.
		\item<2-> The subspace $S_k(\Gamma_0(N),\chi)^\text{new}$ has distinguished elements (the newforms)  which are mutually orthogonal and are eigenvectors for all Hecke operators.
	\end{enumerate}
}

\section{Theta Series}
\subsection{The simplest example}
\frame{
	\frametitle{A half-integral weight theta series}
	Consider the function
	\[\theta(z) = \sum_{x\in\Z}q^{x^2}=1+2q+2q^4+O(q^5).\]
	Then
	\[\theta(\gamma z)=\epsilon(cz+d)^{1/2}\theta(z),\]
	for all $\gamma\in\Gamma_0(4)$ and some $\epsilon_{c,d}\in\lbrace\pm1,\pm i\rbrace$.
}

\subsection{Theta series attached to imaginary quadratic fields}
\frame{
	\frametitle{Theta series attached to ideals}
	Let $K$ be an imaginary quadratic field of discriminant $D<-4$ and let $\O_K$ be its ring of integers. Fix an integer $\ell\geq0$. To each integral ideal $\ida$ of $K$, one can attach the following theta series:
	\[\tha^{(2\ell)}=\tha=\sum_{x\in\ida}x^{2\ell}q^{N(x)/N(\ida)}.\]
	
}

\frame{
	\frametitle{Basic properties of these theta series}
	\begin{enumerate}
		\item We have
		\[\tha=\sum_{x\in\ida}x^{2\ell}q^{N(x)/N(\ida)}\in M_{2\ell+1}(\Gamma_0(|D|),\chi_D),\]
		where $\chi_D$ is the Kronecker symbol. If $\ell\neq0$, then
		\[\tha\in S_{2\ell+1}(\Gamma_0(|D|),\chi_D).\]
		\item If $\lambda\in K^\times$, then
		\[\theta_{\lambda\ida}=\lambda^{2\ell}\tha.\]
		So there are essentially $h_D$ theta series attached to $K$.
		\item In general, the $\tha$ are \emph{not} newforms.
	\end{enumerate}
}

\frame{
	\frametitle{Theta series attached to Hecke characters of $K$}
	Let $I_K$ denote the group of fractionnal ideals of $K$. A Hecke character $\psi$ of $K$ of infinity type $2\ell$ (and conductor $1$) is a homomorphism
	\[\psi:I_K\longrightarrow \C^\times\]
	such that
	\[\psi((\alpha))=\alpha^{2\ell},\hspace{1cm}\forall\alpha\in K^\times.\]
	One can define
	\[\tpsi=\sum_{\ida\subseteq\O_K}\psi(\ida)q^{N(\ida)}.\]
}

\frame{
	\frametitle{Basic properties of these theta series}
	\begin{enumerate}		
		\item We have
		\[\tpsi M_{2\ell+1}(\Gamma_0(|D|),\chi_D),\]
		where $\chi_D$ is the Kronecker symbol. If $\psi^2\neq 1$, then
		\[\tpsi\in S_{2\ell+1}(\Gamma_0(|D|),\chi_D).\]
		\item The $\tpsi$ are newforms.
		\item We have the identities
		\[\tpsi=\frac{1}{w_K}\sum_{[\ida]\in\ClK}\psi^{-1}(\ida)\tha\hspace{0.5cm}\text{and}\hspace{0.5cm}\tha=\frac{w_K}{h_K}\sum_{\psi}\psi(\ida)\tpsi.\]
	\end{enumerate}
}

\subsection{Some questions}
\frame{
	\frametitle{Some questions}
	\begin{itemize}
		\item Can we efficiently compute the Petersson inner product of theta series (whenever it makes sense)?
		\item Can we find explicit formulas for it?
	
		\item Can we use those formulas/computations to study the arithmetic properties of those quantities?
		
		\item What about the $p$-adic properties of these quantities?
	\end{itemize}
}

\section{Explicit formulas}
\subsection{The case $\ell>0$}
\frame{
	\frametitle{Petersson norm of the $\tpsi$ (with $\ell>0$)}
	\begin{theorem}
		Let $\psi$ be a Hecke character of $K$ of infinity type $2\ell$, where $\ell>0$. Then
		\[\langle\tpsi,\tpsi\rangle = V_D^{-1}(|D|/4)^\ell\frac{4h_K}{w_K^2}\sum_{[\ida]\in\ClK}\psi^2(\ida)\partial^{2\ell-1}E_2(\ida),\]
		where
		\[V_D=\text{Vol}(\Gamma_0(|D|)\setminus\H).\]
	\end{theorem}
	Here,
	\[\partial f=\frac{1}{2\pi i}\frac{\partial f}{\partial z}-\frac{k}{4\pi\Im(z)}f\]
	is the Shimura-Mass diffential operator, which preserves the graded algebra of almost holomorphic modular forms.
}

\frame{
	\frametitle{Petersson inner product of the theta series $\tha$}
	\begin{theorem}
		Let $\ida$ and $\idb$ be ideals of $K$ and suppose $\ell>0$. Then
		\[\langle\tha,\thb\rangle=C_K^{(2\ell)}N(\idb)^{2\ell}\sum_{\ida\idb^{-1}\idc^2=\lambda_\idc\O_K}\lambda_\idc^{2\ell}\partial^{2\ell-1}E_2(\idc),\]
		where
		\[C_K^{(2\ell)}=4V_D^{-1}(|D|/4)^\ell.\]
	\end{theorem}
}

\frame{
	\frametitle{A few direct consequences of the formula}
	\begin{coro}
		For $\ell>0$,
		\[\langle\tha,\thb\rangle=0\]
		whenever $\ida$ and $\idb$ are not in the same genus (i.e. the classes of $\ida$ and $\idb$ are distinct in the genus group $\ClK/\ClK^2$).
	\end{coro}
	\begin{coro}
		For $\ell>0$,
		\[\langle\theta_{\ida\idc},\theta_{\idb\idc}\rangle=N(\idb\idc)^{2\ell}\langle\tha,\thb\rangle.\]
	\end{coro}
}

\frame{
	\frametitle{Arithmetic consequences}
	Let
	\[\Omega_K=\frac{1}{\sqrt{4\pi|D|}}\left(\prod_{j=1}^{|D|-1}\Gamma\left(\frac{j}{|D|}\right)\right)^{w_K/4h_k}\]
	be the Chowla-Selberg period attached to $K$.
	\begin{coro}
		For $\ell>0$, the complex numbers
		\[\frac{V_D\langle\tpsi,\tpsi\rangle}{\Omega_K^{4\ell}}\hspace{0.5cm}\text{and}\hspace{0.5cm}\frac{V_D\langle\tha,\thb\rangle}{\Omega_K^{4\ell}}\]
		are algebraic.
	\end{coro}
}

\subsection{The case $\ell=0$}
\frame{
	If $\ell=0$, the modular form $\tha$ is not a cusp form. But for $\tpsi$, we have the following
	\begin{theorem}
		Let $\tpsi$ be a Hecke character of infinity type $0$ and suppose that $\psi^2\neq 1$. Then
		\[\langle\tpsi,\tpsi\rangle = -V_D^{-1}\frac{4h_K}{w_K^2}\sum_{[\ida]\in\ClK}\psi^2(\ida)\log(\Im(\tau_\ida)^{1/2}|\eta(\tau_\ida)|^2),\]
		where $\tau_\ida\in\H$ is the complex root attached to $\ida$ and
		\[\eta(z)=exp(2\pi i/24)\prod_{n=1}^\infty(1-q^n)\]
		is the standard eta-function.
	\end{theorem}
}

\section{Numerical computations}
\subsection{Towards an algorithm}
\frame{
	\frametitle{First step: compute $\partial^nE_2$}
	We have the following formulas:
	\[\partial E_2=\frac{5}{6}E_4-2E_2^2 \hspace{0.5cm}\partial E_4 = \frac{7}{10}E_6-8E_2E_4\hspace{0.5cm}\partial E_6 = \frac{400}{7}E_4^2-12E_2E_6.\]
	For example,
	\[\partial^3E_2=-48E_2^4 + 120E_4E_2^2 - 14E_6E_2 + 25E_4^2.\]
}

\frame{
	\frametitle{Second step: Evaluate Hecke characters}
	The idea is simple: let $\ida$ be a fractional ideal of $K$ and suppose
	\[\ida^e=\lambda\O_K.\]
	Then
	\[\psi(\ida)^e=\psi(\ida^e)=\psi((\lambda))=\lambda^{2\ell},\]
	so $\psi(\ida)$ is determined (up to a $e$-root of unity).
}

\frame{
	\frametitle{Other second step}
	Given ideals $\ida$ and $\idb$, can we efficiently find all classes $[\idc]$ such that
	\[\ida\idb^{-1}\idc^2=\lambda_\idc\O_K,\]
	if any? If we have representatives $\lbrace\ida_1,\dots,\ida_d\rbrace$ of $\ClK[2]$, it suffices to find one such $\idc_0$. Then the other solutions to the equation are
	\[\idc_0\ida_i\]
	for $i=1,\dots,d$.
}

\subsection{Examples of computations}
\frame{
	\frametitle{Class number $1$}
	In this case,
	\[\theta_{\O_K}=\theta_{\psi_0}\]
	and we only need to compute
	\[V_D\langle\theta_{\O_K},\theta_{\O_K}\rangle/\Omega_K^{4\ell}\in\overline{\Q}.\]
}

\frame{
	\frametitle{Class number $1$ case}
	Computation of $V_D\langle\theta_{\O_K},\theta_{\O_K}\rangle/\Omega_K^{4\ell}$:
	\renewcommand{\arraystretch}{1.25}
	\begin{tabular}{cc|*{2}{l|}}
	\cline{3-4}
	& & \multicolumn{2}{ c| }{$\ell$} \\ \cline{3-4}
	& & 1 & 2 \\ \cline{1-4}
	\multicolumn{1}{ |c| }{\multirow{7}{*}{$D$}} & \multicolumn{1}{ |c| }{-7}
	&$2^{2}3$&$-2^{2}$\\
	\cline{2-4}
	\multicolumn{1}{ |c| }{}&\multicolumn{1}{ |c| }{-8}
	&$-2$&$-2^{2}5$\\
	\cline{2-4}
	\multicolumn{1}{ |c| }{}&\multicolumn{1}{ |c| }{-11}
	&$-2^{2}$&$-2^{3}5$\\
	\cline{2-4}
	\multicolumn{1}{ |c| }{}&\multicolumn{1}{ |c| }{-19}
	&$-2^{2}3^{-1}13$&$-2^{3}71$\\
	\cline{2-4}
	\multicolumn{1}{ |c| }{}&\multicolumn{1}{ |c| }{-43}
	&$-2^{3}3^{-1}107$&$-2^{4}5647$\\
	\cline{2-4}
	\multicolumn{1}{ |c| }{}&\multicolumn{1}{ |c| }{-67}
	&$-2^{2}3^{-1}7^{2}31$&$-2^{3}5\cdot86629$\\
	\cline{2-4}
	\multicolumn{1}{ |c| }{}&\multicolumn{1}{ |c| }{-163}
	&$-2^{3}3^{-1}150473$&$-2^{4}11\cdot461681471$\\
	\cline{1-4}
	\end{tabular}
}

\frame{
	\frametitle{Class number $2$}
	In this case, $K$ has two genera. If $\ida$ is a representative of the non-trivial class in $\ClK$, we have
	\[\langle\theta_\ida,\theta_{\O_K}\rangle = \langle\theta_{\O_K},\theta_\ida\rangle=0\]
	and
	\[\langle\theta_\ida,\theta_\ida\rangle=N(\ida)^{2\ell}\langle\theta_{\O_K},\theta_{\O_K}\rangle,\]
	so it suffices to compute the quantity
	\[V_D\langle\theta_{\O_K},\theta_{\O_K}\rangle/\Omega_K^{4\ell}\in\overline{\Q}.\]
}

\frame{
	\frametitle{Class number $2$}
	As in the class number $1$ case, the quantity
	\[V_D\langle\theta_{\O_K},\theta_{\O_K}\rangle/\Omega_K^{4\ell}\]
	is an integer, except for $\ell=1$ and $D=-91,-403$ and $-427$.
}

\frame{
	\frametitle{Class number $3$}
	
}

\subsection{How efficient is this algorithm?}



\section{Idea of the proof}
\end{document}
