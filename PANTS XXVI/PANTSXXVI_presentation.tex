\documentclass{beamer}
\usetheme{Singapore}

% --------------------------------------------------------------------
% Packages pour la couleur
\usepackage{color}

% --------------------------------------------------------------------
% Packages pour les accents français.
\usepackage[latin1]{inputenc}	% L'option "latin1" désigne l'encodage ISO-8859-1, typique sous Linux.
\usepackage[T1]{fontenc}	% Pour les accents
\usepackage[frenchb]{babel}

% --------------------------------------------------------------------
% Voici certains packages souvent utilisés.
\usepackage{graphicx}		% Importation du package permettant d'inclure d'images dans le document.
\usepackage{amsmath, amsfonts}	% Pour écrire selon les standards de l'AMS.
\usepackage{epstopdf}		% Package permettant l'inclusion d'images eps, pour la compilation en pdf.
\usepackage{epsfig}
\usepackage{hyperref}
%\usepackage{subfig}

% --------------------------------------------------------------------
% Le package "palatino" charge la police Palatino en mode texte et le package "euler" charge la police Euler en mode mathématique. Pour retrouver les polices par défaut, effacez les deux lignes de commandes qui suivent.
\usepackage{palatino}
\usepackage{euler}

% --------------------------------------------------------------------
% --------------------------------------------------------------------
% Commande rapide
\newcommand{\C}{\mathbb{C}}
\newcommand{\F}{\mathbb{F}}
\newcommand{\R}{\mathbb{R}}
\newcommand{\Q}{\mathbb{Q}}
\newcommand{\N}{\mathbb{N}}
\newcommand{\Z}{\mathbb{Z}}
\renewcommand{\H}{\mathcal{H}}

\newcommand{\abcdmat}{\begin{pmatrix}
a & b \\ 
c & d
\end{pmatrix}}

\newtheorem{thm}{Theorem}
\newtheorem{defn}[thm]{Definition}
\newtheorem{prop}[thm]{Proposition}
\newtheorem{coro}[thm]{Corollary}
\newtheorem{lem}[thm]{Lemma}
\newtheorem{conj}[thm]{Conjecture}
\newtheorem{rem}[thm]{Remark}
\newtheorem{ex}[thm]{Example}

%Groupe modulo
\newcommand*{\modulo}[2]
{\raisebox{.6ex}{ \newline \ensuremath{#1}}\!/\!\raisebox{-.6ex}{\ensuremath{#2}}}

% --------------------------------------------------------------------
% --------------------------------------------------------------------
% --------------------------------------------------------------------

\title{Petersson Inner Product of Binary Theta Series}
\subtitle{A computational approach}
\author{Nicolas \textsc{Simard}}
\institute{McGill University}
\date{September 17th, 2016}


\begin{document}

\begin{frame}
\titlepage
\end{frame}

\AtBeginSection[]
{
  \begin{frame}
    \frametitle{Table of Contents}
    \tableofcontents[currentsection]
  \end{frame}
}

\section{Background and setup}
\subsection{Modular forms}

\begin{frame}
\frametitle{Mobius transformations}
Let $\H$ be the Poincarre upper-half plane. Recall that $\text{GL}_2(\R)_+$ acts on $\H$ via Mobius transformations:
\[\abcdmat z=\frac{az+b}{cz+d}.\]
\begin{defn}
	Let $N\geq1$ and define the Hecke subgroup of level $N$ as
	\[\Gamma_0(N)=\left\lbrace\abcdmat \in\text{SL}_2(\Z)| c\equiv 0\pmod N\right\rbrace.\]
\end{defn}
	
\end{frame}

\begin{frame}
\frametitle{Level $N$ modular forms with characters}

\begin{defn}
	Let $N\geq1$ and $k\geq0$ be integers and let $\chi$ be a Dirichlet character mod $N$. A modular form of weight $k$, level $N$ and character $\chi$ is a holomorphic function
	\[f:\H\longrightarrow\C\]
	such that
	\[f\left(\gamma z\right)=\chi(d)(cz+d)^{-k}f(z)\]
	for all $z\in\H$ and all $\gamma\in\Gamma_0(N)$, which satisfies certain growth conditions at the cusps.	The $\C$-vector-space of such modular forms is denoted
	\[M_k(\Gamma_0(N),\chi).\]
\end{defn}
	
\end{frame}	
	
\frame{
	\frametitle{$q$-expansion of modular forms}
	Every modular form $f$ has a Taylor (or Fourrier) expansion at infinity, called its $q$-expansion:
	\[f(z)=\sum_{n=0}^\infty a_nq^n,\]
	where $q=exp(2\pi iz)$. If
	\[a_0(f)=0,\]
	$f$ is called a \emph{cusp form}.
}

\frame{
	\frametitle{Example: weight $k$ Eisenstein series}
	Let $k\geq4$ be an even integer and define
	\[G_k(z)=\sum_{m,n}\frac{1}{(mz+n)^k}\in M_k(\Gamma_0(1),1).\]
	After renormalisation, the $q$-expansion of $G_k$ is
	\[E_k(z) = -\frac{B_k}{2k}+\sum_{n=1}^\infty\sigma_{k-1}(n)q^n.\]
}

\frame{
	\frametitle{Important non-example: weight $2$ Eisenstein series}
	In level $1$, there are no modular forms of weight $2$. However, one can still define the weight $2$ Eisenstein series as
	\[E_2(2)=\frac{1}{8\pi\Im(z)}-\frac{1}{24}+\sum_{n=1}^\infty\sigma(n)q^n.\]
	It is an example of an \emph{almost holomorphic} modular form of level $1$ and weight $2$.
}

\subsection{Spaces of modular forms}
\frame{
	\frametitle{Spaces of modular forms}
	\begin{itemize}
		\item<1-> $M_k(\Gamma_0(N),\chi)$ is finite dimensional.
		\item<2-> For every integer $n\geq1$, one can define a \emph{Hecke operator} $T_n$ (depending on $k$, $N$ and $\chi$) which acts on $M_k(\Gamma_0(N),\chi)$.
		\item<3-> There exists a basis of common eigenvectors for all Hecke operators $T_n$ with $(n,N)=1$.
	\end{itemize}
}

\frame{
	\frametitle{Petersson inner product}
	Let $f,g\in S_k(\Gamma_0(N),\chi)$ be two cusp forms. The Petersson inner product of $f$ and $g$ is defined as
	\[\langle f,g\rangle =\frac{1}{\text{Vol}(\Gamma_0(N)\setminus\H)}\int_{\Gamma_0(N)\setminus\H}f(x+iy)\overline{g(x+iy)}y^k\text{d}\mu,\]
	where
	\[\text{d}\mu=\frac{\text{d}x\text{d}y}{y^2}\]
	is the $\text{SL}_2(\R)$-invariant measure on $\H$.
	Note that the intergal does not converge if neither $f$ nor $g$ is a cusp form.
}

\subsection{Newforms}
\frame{
	\frametitle{Newforms}
	The space $S_k(\Gamma_0(N),\chi)$ splits naturally as
	\[S_k(\Gamma_0(N),\chi)=S_k(\Gamma_0(N),\chi)^{\text{new}}\oplus S_k(\Gamma_0(N),\chi)^\text{old}.\]
	\begin{theorem}
		The space $S_k(\Gamma_0(N),\chi)^{\text{new}}$ has an \emph{orthogonal} basis of eigenvectors for \emph{all} Hecke operators. Elements of this basis are called newforms (after suitable normalization).
	\end{theorem}
}

\frame{
	\frametitle{Summary}
	\begin{enumerate}
		\item<1-> The space $S_k(\Gamma_0(N),\chi)$ is a finite dimensional inner product space, equiped with an action of Hecke operators.
		\item<2-> The subspace $S_k(\Gamma_0(N),\chi)^\text{new}$ has distinguished elements (the newforms)  which are mutually orthogonal and are eigenvectors for all Hecke operators.
	\end{enumerate}
}

\section{Theta Series}
\subsection{The simplest example}
\frame{
	\frametitle{A half-integral weight theta series}
	Consider the function
	\[\theta(z) = \sum_{x\in\Z}q^{x^2}=1+2q+2q^4+O(q^5).\]
	Then
	\[\theta(\gamma z)=\epsilon(cz+d)^{1/2}\theta(z),\]
	for all $\gamma\in\Gamma_0(4)$ and some $\epsilon_{c,d}\in\lbrace\pm1,\pm i\rbrace$.
}

\subsection{Theta series attached to imaginary quadratic fields}
\frame{
	\frametitle{The setup for this talk}
	
}


\end{document}
