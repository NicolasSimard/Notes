\documentclass[twoside,10pt]{article}

\usepackage{amsmath, amssymb, amsthm}
\usepackage[top=1in, left=1.25in, right=1in, bottom=1in]{geometry}%Pour des pages plus larges
\renewcommand*\familydefault{\sfdefault}%Pour des lettres sans serif

\usepackage{graphicx}%Pour les images

\usepackage[pdftex,bookmarks,colorlinks,breaklinks]{hyperref} 
\hypersetup{linkcolor=blue,citecolor=red,filecolor=dullmagenta,urlcolor=darkblue}

%\usepackage{palatino}%Pour utiliser l'�criture Palatino
\usepackage{euler}%Change l'aspect des formules math�matiques

\usepackage{multirow}

\usepackage[dvips,matrix,ps,pdf,color,line,graph]{xy}
\input xy
\xyoption{all}

\newtheorem{theorem}{Theorem}
\newtheorem{lemma}{Lemma}
\newtheorem{prop}{Proposition}
\newtheorem{defn}{Definition}
\newtheorem{coro}{Corollary}

\newcommand{\rarr}{\rightarrow}
\newcommand{\esp}{\hspace{1cm}}
\newcommand{\for}{\hspace{1cm}\textrm{for }}
\newcommand{\where}{\hspace{1cm}\textrm{where }}
\newcommand{\myand}{\hspace{1cm}\text{and}\hspace{1cm}}

\newcommand{\Z}{\mathbb{Z}}
\newcommand{\Q}{\mathbb{Q}}
\newcommand{\R}{\mathbb{R}}
\newcommand{\C}{\mathbb{C}}

\newcommand{\T}{\mathfrak{T}}
\newcommand{\B}{\mathfrak{B}}
\newcommand{\Gfrak}{\mathfrak{G}}
\newcommand{\Hfrak}{\mathfrak{H}}
\newcommand{\m}{\mathfrak{m}}
\newcommand{\p}{\mathfrak{p}}
\newcommand{\ida}{\mathfrak{a}}
\newcommand{\idb}{\mathfrak{b}}
\newcommand{\idc}{\mathfrak{c}}
\newcommand{\idd}{\mathfrak{d}}

\newcommand{\curlG}{\mathcal{G}}
\newcommand{\curlF}{\mathcal{F}}
\newcommand{\curlM}{\mathcal{M}}
\newcommand{\curlH}{\mathcal{H}}
\newcommand{\curlY}{\mathcal{Y}}
\newcommand{\curlU}{\mathcal{U}}
\newcommand{\curlK}{\mathcal{K}}
\newcommand{\curlN}{\mathcal{N}}
\newcommand{\curlL}{\mathcal{L}}
\renewcommand{\S}{\mathcal{S}}
\renewcommand{\O}{\mathcal{O}}

\newcommand{\eqdef}{:=}
\newcommand{\Dist}{\mathfrak{D}}
\newcommand{\Meas}{\mathfrak{M}}
\newcommand{\Hom}{\text{Hom}}
\newcommand{\Step}{\text{Step}}
\newcommand{\charf}{\varepsilon}
\newcommand{\Gal}{\text{Gal}}
\renewcommand{\d}{\text{d}}
\newcommand{\Res}{\text{Res}_{\Z_p^\times}}

\author{Nicolas Simard}
\date{\today}
\title{Constructing the p-adic zeta function via cyclotomic units}

\begin{document}
\maketitle
\tableofcontents

\section*{Introduction}

\section{$p$-adic measures}
In this section, we first define $p$-adic measures and see how they are related to Iwasawa Algebras and power series rings. We then introduce operators on them and conclude with of few results on moments of measures.

\subsection{$p$-adic measures, distributions and Iwasawa algebras}
Let $\Gfrak$ be an abelian profinite group, let $\B_\Gfrak$ be the boolean algebra of open subsets of $\Gfrak$, let $\T_\Gfrak\subseteq\B_\Gfrak$ be the set of open subgroups of $\Gfrak$ and let $A$ be any abelian group.

\begin{defn}
	An $A$-valued distribution $\lambda$ on $\Gfrak$ is a finitely additive function
	\[\lambda:\T_\Gfrak\rarr A.\]
	The set of distributions is denoted $\Dist(\Gfrak,A)$. If $A\subseteq\C_p$, the elements of $\Dist(\Gfrak,A)$ are called $p$-adic distributions.
\end{defn}

The set $\Dist(\Gfrak,A)$ is naturally an abelian group. If $A$ is a $B$-algebra for some ring $B$, the set $\Dist(\Gfrak,A)$ is a $B$-algebra under convolution product, which we won't bother to define here!

\subsubsection{Distributions and Iwasawa algebras}
If $\Gfrak$ is finite, $\B_\Gfrak = \{\{g\}|g\in\Gfrak\}$ and we have an isomorphism of abelian groups
\[\lambda\mapsto\sum_{g\in\Gfrak}\lambda(\{g\})g:\Dist(\Gfrak,A)\rightarrow A[\Gfrak].\]
If $A$ is a $B$-algebra for some ring $B$, so is $A[\Gfrak]$ and the isomorphism is an isomorphism of $B$-algebras. For $\Gfrak$ finite, we define
\[\Lambda(\Gfrak,A)\eqdef A[\Gfrak]\]
and call it an Iwasawa algebra.

For $\Gfrak$ not necessarily finite, we define
\[\Lambda(\Gfrak,A)=\varprojlim\Lambda(\Gfrak/\Hfrak,A)=\varprojlim A[\Gfrak/\Hfrak],\]
where the limit is taken over all elements of $\T_\Gfrak$. Given $\Hfrak$ in $\T_\Gfrak$, we have a natural map
\[\lambda\mapsto\sum_{g\mod\Hfrak}\lambda(g)(g+\Hfrak):\Dist(\Gfrak,A)\rarr\Lambda(\Gfrak/\Hfrak,A).\]
Since distributions are finitely additive, we have then a natural map
\[\Dist(\Gfrak,A)\rarr\Lambda(\Gfrak,A),\]
which is in fact an isomorphism. In a certain sense, the elements of the Iwasawa algebra $\Lambda(\Gfrak,A)$ are like the generating series of distributions. 

\textbf{Example: } If $A=\Z_p$, one obtains the usual Iwasawa algebra
\[\Lambda(\Gfrak)\eqdef\Lambda(\Gfrak,\Z_p).\]

\subsubsection{Distributions and step functions}
From now on, suppose that $A$ is a $B$-algebra fro some ring $B$.

Recall that if $s:\Gfrak\rightarrow A$ is a locally constant function, also called a step function, there exists an open subgroup $\Hfrak$ such that $s$ is well defined on $\Gfrak/\Hfrak$, i.e.
\[s(x) = \sum_{g\in\Gfrak\Hfrak}s(g)\charf_{g}(x),\]
where $\charf_{g}(x)$ is the characteristic function of the (open) coset $g\in\Gfrak/\Hfrak$. Note that such a representation of $s$ as a finite linear combination of characteristic functions is not unique. The set of step functions from $\Gfrak$ to $A$ is denoted
\[\Step(\Gfrak,A).\]
If $A$ is a $B$-algebra, $\Step(\Gfrak,A)$ is a $B$-algebra under the pointwise addition and multiplication of functions.

Let $\lambda$ be an $A$-valued distribution on $\Gfrak$, let $s$ be a step function which is well-defined $\mod\Hfrak$ as above and define
\[\int_\Gfrak s\d\lambda\eqdef\sum_{g\in \Gfrak/\Hfrak}s(g)\lambda(g).\]
This gives a well-defined $B$-linear map
\[\Lambda(\Gfrak,A)\rarr\Hom_{B-\text{alg}}(\Step(\Gfrak,A),A).\]
For convenience, the value of an element $\lambda\in\Hom_{B-\text{alg}}(\Step(\Gfrak,A),A)$ at a step function $s(x)$ is denoted
\[\int_{\Gfrak}s(x)\d\lambda(x)\]
or simply
\[\int_{\Gfrak}s\d\lambda\]
when there is no risk of confusion. The $B$-module
\[\Hom_{B-\text{alg}}(\Step(\Gfrak,A),A)\]
can be equipped with a natural $B$-algebra structure via the convolution product which is defined as follows: for $\lambda,\mu\in\Hom_{B-\text{alg}}(\Step(\Gfrak,A),A)$, let $\lambda\ast\mu$ be defined as 
\[\int_{\Gfrak}f\d\lambda\ast\mu=\int_{\Gfrak}\left (f(x+y)\d\lambda(x)\right )\d\mu(y).\]
The map above is then a $B$-algebra homomorphism. In fact, it is an isomorphism. Indeed, its inverse takes a homomorphism
$\lambda\in\Hom_{B-\text{alg}}(\Step(\Gfrak,A),A$ to the distribution $lambda$ defined as
\[\lambda(U)=g(\charf_U).\]
This sketches the proof of the following proposition.

\begin{prop}
	There is a natural $B$-algebra isomorphism
	\[\Lambda(\Gfrak,A)\rarr\Hom_{B-\text{alg}}(\Step(\Gfrak,A),A).\]
\end{prop}

\subsubsection{$p$-adic measures and continuous functions}
From now on, suppose that $A$ is contained in $\C_p$ (e.g. $A=B=\Z_p$). Let
\[C(\Gfrak,\C_p)\]
be the set of continuous functions from $\Gfrak$ to $\C_p$. This is a $\C_p$-Banach when equipped with the norm
\[\parallel f\parallel=\sup_{x\in\Gfrak} |f(x)|_p.\]
The set $\Step(\Gfrak,\C_p)$ is dense in $C(\Gfrak,\C_p)$.

\begin{defn}
	A $p$-adic distribution $\lambda\in\Dist(\Gfrak,A)$ is called a $p$-adic measure if it is bounded (as a function from $\T_\Gfrak$ to $A\subseteq\C_p$). The set of $p$-adic measures is denoted $\Meas(\Gfrak,A)$.
\end{defn}

Note that if $A$ is bounded, which is the case if $A=\Z_p$ for example, then $\Meas(\Gfrak,A)=\Dist(\Gfrak,A)$. The importance of introducing measures is the following proposition.

\begin{prop}
	Let $\lambda\in\Meas(\Gfrak,\C_p)$ be viewed as an element of
	\[\Hom_{B-\text{alg}}(\Step(\Gfrak,\C_p),\C_p)\]
	via the above isomorphism. Then $\lambda$ extends uniquely to a continuous map
	\[\lambda:C(\Gfrak,\C_p)\rightarrow\C_p.\]
\end{prop}
\begin{proof}
	Let $\lambda$ be a $p$-adic measure and suppose that
	\[|\lambda(U)|_p\leq M\]
	for all $U\in\B_\Gfrak$ and some $M\in\R$. By the density of $\Step(\Gfrak,\C_p)$ in $C(\Gfrak,\C_p$, for any $f\in C(\Gfrak,\C_p)$ one can find a sequence of step functions $s_n\in\Step(\Gfrak,\C_p)$ such that
	\[f(x)=\lim_{n\rightarrow\infty} s_n(x).\]
	Then it is easy to see that for any integers $m$ and $n$,
	\[\lambda(s_n-s_m)\leq M\parallel s_n-s_m\parallel.\]
	Since the sequence $\{s_n\}$ is Cauchy, so is the sequence $\{\lambda(s_n)\}$ and is makes sense to define
	\[\lambda(f)=\lim_{n\rightarrow\infty} \lambda(s_n).\]
	The uniqueness is clear.
\end{proof}

For $\lambda\in \Hom_{\text{cont}}(C(\Gfrak,\C_p),\C_p)$, define
\[\parallel\lambda\parallel=\sup_{f\in C(\Gfrak,\C_p)}\frac{|\lambda(f)|}{\parallel f\parallel},\]
which is a finite real number by the continuity of $\lambda$. Equipped with the convolution product, this set becomes a $\C_p$-Banach algebra.

In the case where $A=\Z_p$, we have $\Meas(\Gfrak,\Z_p)=\Dist(\Gfrak,\Z_p)$ and we have the following proposition.

\begin{prop}
	The image of $\Meas(\Gfrak,\Z_p)$ under the injection of the previous proposition is the set of
	\[\lambda\in \Hom_{\text{cont}}(C(\Gfrak,\C_p),\C_p)\]
	such that
	\[\parallel\lambda\parallel\leq 1\hspace{1cm}\text{and}\hspace{1cm}\lambda(C(\Gfrak,\Q_p))\subseteq\Q_p.\]
\end{prop}
\begin{proof}
	Let $\lambda\in\Meas(\Gfrak,\Z_p)$ and take
	\[s\in\Step(\Gfrak,\Q_p).\]
	Writing
	\[s=\sum_{g\in\Gfrak/Hfrak}s(g)\charf_g,\]
	we see that
	\[\int_{\Gfrak}s\d\lambda=\sum_{g\in\Gfrak/Hfrak}s(g)\lambda(g)\in\Q_p\]
	and so
	\[\left |\int_{\Gfrak}s\d\lambda\right |_p\leq\sup_{g\in\Gfrak/Hfrak}|s(g)|_p|\lambda(g)|_p\leq\parallel s\parallel.\]
	From the density of $\Step(\Gfrak,\C_p)$ in $C(\Gfrak,\C_p)$ and the continuity of the norm function, it follows that
	\[\parallel\lambda\parallel\leq 1\and\lambda(C(\Gfrak,\Q_p))\subseteq\Q_p.\]
	
	Conversely, let
	\[\lambda\in \Hom_{\text{cont}}(C(\Gfrak,\C_p),\C_p)\]
	be such that
	\[\parallel\lambda\parallel\leq 1\hspace{1cm}\text{and}\hspace{1cm}\lambda(C(\Gfrak,\Q_p))\subseteq\Q_p.\]
	Then
	\[\lambda(\charf_U)\in\Q_p\]
	for any $U\in\B_\Gfrak$ since $\charf_U\in C(\Gfrak,\Q_p)$. Moreover,
	\[\parallel\charf_U\parallel=1,\]
	and $\parallel\lambda\parallel\leq 1$, so in fact
	\[\lambda(\charf_U)\in\Z_p.\]
	This concludes the proof.	
\end{proof}

If $\rho:\Gfrak\rarr\C_p^\times$ is a character, i.e. a group homomorphism, and $\lambda,\mu\in\Meas(\Gfrak,\C_p)$ then
\[\int_{\Gfrak}\rho\d\lambda\ast\mu=\int_\Gfrak\rho\d\lambda\int_\Gfrak\rho\d\mu.\]
A \emph{pseudo-measure} is an element $\lambda$ of the total ring of fractions of $\Lambda(\Gfrak)$, i.e. a quotient $\lambda=\mu/\nu$ of elements $\Lambda(\Gfrak,\Z_p)$ where $\nu$ is not a zero divisor, with the property that
\[(g-1)\lambda\in\Lambda(\Gfrak).\]
For any pseudo-measure $\lambda$ as above and any non-trivial character $\rho$ of $\Gfrak$, define
\[\int_\Gfrak\rho\d\lambda\eqdef\frac{\int_\Gfrak\rho\d\mu}{\int_\Gfrak\rho\d(g-1)}=\frac{\int_\Gfrak\rho\d\mu}{\rho(g)-1},\]
where $g$ is any element of $\Gfrak$ not in the kernel of $\rho$. This definition does not depend on this choice of $g$. Note that we used the fact that for any $g\in\Gfrak$,
\[\int_\Gfrak f\d g=f(g).\]
In other words, the elements of $\Gfrak$ correspond to Dirac measures.

\subsubsection{The Iwasawa algebra $\Lambda(\Z_p)$ and Mahler's transform}
When $\Gfrak=\Z_p$, one can say more about $p$-adic measures. This is because the $\C_p$-Banach algebra of continuous functions on $\Z_p$ has a special \emph{Mahler basis}.

For $n\in\Z_{\geq0}$, define
\[e_n(x)\eqdef\binom{x}{n}=\frac{x(x-1)\dots(x-n+1)}{n!}.\]

\begin{theorem}
	Let $f\in C(\Z_p,\C_p)$. Then there exists a unique sequence $\{a_n\}_{n\geq 0}$ of elements of $\C_p$ such that
	\[\lim_{n\rightarrow\infty}a_n=0\]
	and
	\[f(x)=\sum_{n=0}^\infty a_n\binom{x}{n}.\]
	This is called the Mahler expansion of $f$.
\end{theorem}
\begin{proof}
	This is Theorem $3.3.1$ in \cite{CS}.
\end{proof}

Knowing that elements of $\Lambda(\Z_p)$ can be viewed as continuous linear functional on $ C(\Z_p,\C_p)$, one could form their generating function with respect to the Mahler basis:
\[\curlM(\lambda)\eqdef\sum_{n=0}^\infty T^n\int_{\Z_p}\binom{x}{n}\d\lambda.\]
This is called the \emph{Mahler transform} of $\lambda$. Note that
\[\curlM(\lambda)\in\Z_p[[T]].\]
Intuitively, the Mahler transform should determine $\lambda$ (because the $e_n(x)$ form a basis of $C(\Z_p,\C_p)$). In fact, much more is true.

\begin{theorem}
	The Mahler transform
	\[\curlM:\Lambda(\Z_p)\rarr \Z_p[[T]]\]	
	is an isomorphism of $\Z_p$-algebras.
\end{theorem}
\begin{proof}
	This is Theorem $3.3.3$ in \cite{CS}.
\end{proof}

The inverse of $\curlM$, denoted $\curlY$ in \cite{CS}, is defined as follows. If a continuous function $f$ has Mahler expansion
\[f(x)=\sum_{n=0}^\infty a_n e_n(x)\]
and
\[g(T)=\sum_{n=0}^\infty b_nT^n\in\Z_p[[T]],\]
we define
\[\int_{\Z_p} f\d\curlY(g)=\sum_{n=0}^\infty a_nb_n.\]
For convenience, we sometimes denote $\curlY(g)$ by $\lambda_g$.

\textbf{Example: } For any $a\in\Z_p$, viewed as a constant compatible sequence in $\Lambda(\Z_p)$, one has
\[\curlM(a)=(1+T)^a,\]
so that the power series $(1+T)^a$ corresponds to the Dirac measures in $\Lambda(\Z_p)$.

\subsubsection{The Iwasawa algebra $\Lambda(\Z_p^\times)$}
Integration over $\Gfrak=\Z_p^\times$ is closely related to integration over $\Z_p$. Since $\Lambda(\Z_p)$ has more structure, it is desirable to relate $\Lambda(\Z_p^\times)$ to $\Lambda(\Z_p)$. Since $\Z_p^\times$ is a subset of $\Z_p$, it is natural to define a map
\[\imath:\Lambda(\Z_p^\times)\rarr\Lambda(\Z_p)\]
as
\[\int_{\Z_p}f\d\imath(\lambda)\eqdef\int_{\Z_p^\times}f|_{\Z_p^\times}\d\lambda,\]
for all $f\in C(\Z_p,\C_p)$, where $f|_{\Z_p^\times}\in C(\Z_p^\times,\C_p)$ is the restriction of $f$ to $\Z_p^\times$. One can check that this map is well-defined, i.e. that the functional 
\[f\mapsto \int_{\Z_p}f\d\iota(\lambda)\]
is in the image of $\Lambda(\Z_p)$ in $\lambda\in \Hom_{\text{cont}}(C(\Gfrak,\C_p),\C_p)$.

The next step is to identify the image of $\Lambda(\Z_p^\times)$ inside $\Lambda(\Z_p)$. This will be done in the next section, after we introduce the trace and restriction operators. 

\subsubsection{The Iwasawa algebras $\Lambda(\curlG)$ and $\Lambda(G)$}
Recall the following notation
\[\curlF_n=\Q(\mu_{p^{n+1}})\myand F_n=\Q(\mu_{p^{n+1}})^+.\]
\[\curlG=\Gal(\curlF_\infty/\Q)\myand G=\Gal(F_\infty/\Q).\]
A generator $(\zeta_n)$ for the Tate module $T_p(\mu)$ is a sequence of $\zeta_n\in\mu _{p^{n+1}}$ such that $\zeta_{n+1}^p=\zeta_n$. Fixing such a generator, we obtain an isomorphism
\[\chi:\curlG\rarr\Z_p^\times,\]
called the cyclotomic character. This indices an isomorphism
\[\tilde{\chi}:\Lambda(\curlG)\rarr\Lambda(\Z_p^\times).\]
But more is true. One can define a natural action of $\curlG$ on $\Lambda(\Z_p^\times)$ and $\Lambda(\Z_p)$ via the cyclotomic character. Then $\tilde{\chi}$ becomes a $\curlG$-isomorphism, i.e. $\tilde{\chi}(\curlG)$ is $\curlG$-equivariant.

For each $n\geq0$, the CM field $\curlF_n$ has complex conjugation action $\iota_n$ and the fixed field of $\{1,\iota_n\}$ is $F_n$. This extends to a complex conjugation action $\iota$ in $\curlG$ which fixes $G$. This makes $\Lambda(\curlG)$ into a $\Z_p[\mathcal{J}]$-module, where $\mathcal{J}=\{1,\iota\}$. For $p$ odd decomposes naturally as
\[\Lambda(\curlG)=\Lambda(\curlG)^+\oplus\Lambda(\curlG)^-,\]
where
\[\Lambda(\curlG)^+=\frac{1+\iota}{2}\Lambda(\curlG)\myand\Lambda(\curlG)^-=\frac{1-\iota}{2}\Lambda(\curlG).\]
Finally, one has the following proposition.

\begin{prop}
	The restriction to $\Lambda(\curlG)^+$ of the natural surjection from $\Lambda(\curlG)$ to $\Lambda(G)$ induces an isomorphism
	\[\Lambda(\curlG)^+\simeq\Lambda(G).\]
\end{prop}
\begin{proof}
	This is Lemma $4.2.1$ of \cite{CS}.
\end{proof}

\subsection{Operators on $p$-adic measures}
In \cite{CS}, the authors introduce a few operators in the ring $R=\Z_p[[T]]$. Since $\Lambda(\Z_p)$ is canonically isomorphic to this ring via the Mahler transform, those operators have a corresponding simple definition on the Iwasawa algebra. By combining those operators, one obtains the restriction operator, which has a natural interpretation.

\subsubsection{Operators on $R$}
Let $g(T)$ be a power series in $R$ and define the operator
\[\varphi:R\rarr R\]
as
\[\varphi(g)(T)=g((1+T)^p-1).\]
This is well defined \emph{injective} $\Z_p$-algebra endomorphism (see \cite[Lemma 2.2.2]{CS}).

One can the define trace operator
\[\psi:R\rarr R\]
as
\[(\varphi\circ\psi)(g)(T)=\frac{1}{p}\sum_{\xi\in\mu_p}g(\xi(1+T)-1).\]
This is a well-defined continuous $\Z_p$-linear map from $R$ to itself (see \cite[Proposition 2.2.3]{CS}). Moreover,
\[\psi\circ\varphi=1_R.\]

Finally, one can introduce a derivation $D$ on $R$ as follows:
\[D(g)(T)=(1+T)\frac{\d g}{\d T}.\]

It is enlightening to interpret those operators in a different way. Suppose that $g(T)$ can be written as
\[g(T)\sum_{n=0}^\infty a_n(1+T)^n.\]
Then $\varphi$ is simply given as
\[\varphi(g)(T)=\sum_{n=0}^\infty a_n(1+T)^{pn}.\]
As for $\psi$, as simple calculation shows that
\[\psi(g)(T)=\sum_{n=0}^\infty a_{np}(1+T)^n.\]
Moreover,
\[D(g)(T)=\sum_{n=0}^\infty na_n(1+T)^n.\]
Letting $q=1+T$, this proves that the $\varphi$, $\psi$ and $D$ operators correspond formally to the $V_p$, $U_p$ and $q\frac{\d}{\d q}$ operators on $q$-expansions of level $1$ modular forms. With that in mind, it is clear that $\psi\circ\phi$ is the identity on $R$.

\subsubsection{Operators on $\Lambda(\Z_p)$}
We now introduce the operators on $p$-adic measures, i.e. elements of the Iwasawa algebra $\Lambda(Z_p)$, which correspond to $\varphi$, $\psi$ and $D$ on $R$.

Let $\lambda\in\Lambda(Z_p)$ be a $p$-adic measure on $\Z_p$. Then one can verify  without difficulty that the $\Z_p$-algebra endomorphism
\[\varphi:\Lambda(Z_p)\rarr\Lambda(Z_p)\]
defined as
\[\int_{\Z_p}f(x)\d\varphi(\lambda)(x)\eqdef\int_{\Z_p}f(px)\d\lambda(x)\]
corresponds, via the Mahler transform, to the operator $\varphi:R\rarr R$ introduced above.

A similar calculation shows that the $\Z_p$-linear map
\[\psi:\Lambda(Z_p)\rarr\Lambda(Z_p)\]
defined as
\[\int_{\Z_p}f(x)\d\psi(\lambda)(x)\eqdef\int_{\Z_p}\charf_{p\Z_p}(x)f\left (\frac{x}{p}\right )\d\lambda(x)\]
corresponds to the $\Z_p$-linear map $\psi:R\rarr R$ introduced above.

One can then see, directly or using the corresponding property on $R$, that
\[\psi\circ\varphi=1_{\Lambda(\Z_p)}.\]
One also sees that $\varphi\circ\psi$ corresponds to "restriction on $p\Z_p$", since
\[\int_{\Z_p}f(x)\d(\varphi\circ\psi)(\lambda)(x)=\int_{\Z_p}f(px)\d\psi(\lambda)(x)=\int_{\Z_p}\charf_{p\Z_p}(x)f(x)\d\lambda(x).\]

Now let $f_0(x)$ be any continuous function on $\Z_p$ and define the measure $f_0\lambda$ as
\[\int_{\Z_p}f(x)\d(f_0\lambda)(x)=\int_{\Z_p}f_0(x)f(x)\d\lambda(x).\]
For $f_0(x)=x$, one has the relation
\[\curlM(x\lambda)=D(\curlM(\lambda)),\]
which follows formally from the identity
\[x\binom{x}{n}=(n+1)\binom{x}{n+1}+n\binom{x}{n}\]
(see the proof of Lemma $3.3.5$ in \cite{CS}). Therefore the $D$ operator corresponds to the multiplication by $x$ map on $\Lambda(\Z_p)$.

\subsubsection{Restriction of measures from $\Z_p$ to $\Z_p^\times$}
We now introduce the restriction operator. In particular, this will allow us to identify the image of $\Lambda(\Z_p^\times)$ inside $\Lambda(\Z_p)$.

Recall that the operator $\delta:R\rarr R$ is defined in section $3.4$ of \cite{CS} as
\[\delta(g)(T)=g(T)-\varphi\circ\psi(g)(T)=(1-\varphi\circ\psi)(g)(T).\]
We define the restriction operator as
\[\Res\eqdef 1-\varphi\circ\psi.\]
It is not so clear why this operator on power series should be viewed as a restriction operator. However, on measures we have
\begin{align*}
	\int_{\Z_p}f(x)\d\Res(\lambda)(x) 	&= \int_{\Z_p}f(x)\d\lambda(x)-\int_{\Z_p}f(x)\d(\varphi\circ\psi)(\lambda)(x)\\																&= \int_{\Z_p}f(x)\d\lambda(x)-\int_{\Z_p}\charf_{p\Z_p}(x)f(x)\d\lambda(x)\\
										&= \int_{\Z_p}(1-\charf_{p\Z_p}(x))f(x)\d\lambda(x)\\
										&= \int_{\Z_p}\charf_{\Z_p^\times}(x)f(x)\d\lambda(x)\\
\end{align*}
which justifies the notation. Note that the operator $\Res$ on measures is denoted $\#$ in \cite{CS}.

The operator $\Res$ is a projection, i.e. $\Res\circ\Res=\Res$. A formal computation shows that
\[\Res g(T)=g(T) \Leftrightarrow \psi(g)(T)=0\Leftrightarrow g\in R^{\psi=0},\]
where
\[R^{\psi=0}=\{g\in R|\psi(g)=0\}.\]

\begin{prop}
	The image of $\Lambda(\Z_p^\times)$ in $\Lambda(\Z_p)$ under the injection $\imath$ is the set of measures fixed by the $\Res$ operator. 
\end{prop}
\begin{proof}
	This follows from Lemma $3.4.1$ and Lemma $3.4.2$ in \cite{CS}.
\end{proof}

This proposition means that the restriction of $p$-adic measures on $\Z_p$ can be viewed as $p$-adic measures on $\Z_p^
times$. It also implies that the following diagram
\[\xymatrix{
	\Lambda(\Z_p)\ar[r]^{\curlM} & R\\
	\Lambda(\Z_p^\times)\ar[u]^{\imath}\ar[r]^{\curlM\circ\imath} & R^{\psi=0}\ar[u]
}\]
is commutative.

Using the analogy between $\varphi\leftrightarrow V_p$ and $\psi\leftrightarrow U_p$ discussed above, we see that the restriction operator looks like the $p$-stabilisation operator on modular forms. 

\subsection{Moments of $p$-adic measures}
The special values of the zeta function will be obtained by computing the moments of a pseudo-measure on $\Lambda(\curlG)$. We collect here a few results that help us compute those moments later.

First, it is follows directly from the results of the previous section that
\[\int_{\Z_p}x^k\d\lambda(x)=\int_{\Z_p}\d((x^k\lambda)(x))=\curlM(x^k\lambda)(0)\]
and since
\[\curlM(x\lambda)=D\curlM(\lambda)\]
we have
\begin{equation}\label{eq:momentsandD}
	\int_{\Z_p}x^k\d\lambda(x)=D^k\curlM(\lambda)(0).
\end{equation}


Second, one would like to have a relation between
\[\int_{\Z_p}x^k\d\lambda(x)\myand \int_{\Z_p}x^k\d(\Res\lambda)(x).\]
To have a simple relation, \emph{suppose $\psi(\lambda)=\lambda$}. We compute
\begin{align*}
	\int_{\Z_p}x^k\d\Res(\lambda)(x) 	&= \int_{\Z_p}x^k\d(1-\varphi\circ\psi)(\lambda)(x) &\\																							&= \int_{\Z_p}x^k\d(1-\varphi)(\lambda)(x)			&\text{since $\psi(\lambda)=\lambda$}\\
										&= \int_{\Z_p}x^k\d\lambda(x)-\int_{\Z_p}x^k\d\varphi(\lambda)(x) &\\
										&= \int_{\Z_p}x^k\d\lambda(x)-\int_{\Z_p}(px)^k\d\psi(\lambda)(x) &\\
										&= (1-p^k)\int_{\Z_p}x^k\d\lambda(x). &\\
\end{align*}
In brief,
\begin{equation}\label{eq:momentsofres}
	\int_{\Z_p}x^k\d\Res(\lambda)(x)=(1-p^k)\int_{\Z_p}x^k\d\lambda(x).
\end{equation}

Note that this is consistent with our observation that the restriction operator can be thought of as a $p$-stabilisation operator, since multiplication by $1-p^k$ corresponds to $p$-stabilisation on $L$-functions (i.e. removing the euler factors at $p$).

Finally, moments determine measures on $\Z_p^\times$.

\begin{prop}
	Let $\lambda\in\Lambda(\curlG)$ be a measure. If
	\[\int_\curlG\chi^k(g)\d\lambda(g)\esp\text{for }k=1,3,5,\dots,\]
	then $\lambda\in\Lambda(\curlG)^+$. Similarly, if
	\[\int_\curlG\chi^k(g)\d\lambda(g)\esp\text{for }k=2,4,6,\dots,\]
	then $\lambda\in\Lambda(\curlG)^-$. In particular,
	\[\int_\curlG\chi^k(g)\d\lambda(g)\esp\text{for all }k>0,\]
	then $\lambda=0$. Similar statements are true for pseudo-measures.
\end{prop}
\begin{proof}
	This is Lemma $4.4.2$ and Corollary $4.2.3$ of \cite{CS}.
\end{proof}

\subsection{Summary}
\begin{enumerate}
\item For a general abelian group, we have
\[\Dist(\Gfrak,A)\simeq\Lambda(\Gfrak,A)\]
canonically as abelian groups. When $A$ is a $B$-algebra, this is an isomorphism of $B$-algebras.

\item When $A$ is a $B$-algebra, we have
\[\Lambda(\Gfrak,A)\simeq\Hom_{B-\text{alg}}(\Step(\Gfrak,A),A),\]
canonically as $B$-algebras, where the product on the right set is the convolution product.

\item When $A$ is contained in $\C_p$, any $p$-adic measure $\lambda$ extends uniquely to a continuous map
\[\lambda:C(\Gfrak,\C_p)\rightarrow\C_p.\]

\item When $A=\Z_p$, the image of $\Meas(\Gfrak,\Z_p)$ under the above injection is the set of
	\[\lambda\in \Hom_{\text{cont}}(C(\Gfrak,\C_p),\C_p)\]
	such that
	\[\parallel\lambda\parallel\leq 1\hspace{1cm}\text{and}\hspace{1cm}\lambda(C(\Gfrak,\Q_p))\subseteq\Q_p.\]
	
\item When $A=\Z_p$ and $\Gfrak=\Z_p$, the Mahler transform $\curlM$ establishes a $\Z_p$-algebra isomorphism
\[\curlM:\Lambda(\Z_p)\rarr\Z_p[[T]].\]

\item When $A=\Z_p$ and $\Gfrak=\Z_p^\times$, we have a natural injection
\[\iota:\Lambda(\Z_p^\times)\rarr\Lambda(\Z_p).\]

\item When $A=\Z_p$ and $\Gfrak=\curlG$, we have a $\curlG$-isomorphism $\Lambda(\curlG)\approx\Lambda(\Z_p^\times)$. Moreover, $\Lambda(G)$ can be canonically identified as a $\Z_p$-submodule of $\Lambda(\curlG)$.
\end{enumerate}

Here are all the maps we introduced for $\Z_p$-valued measures:
\[\Lambda(G)\simeq\Lambda(\curlG)^+\hookrightarrow\Lambda(\curlG)\overset{\tilde{\chi}}{\approx}\Lambda(\Z_p^\times)\overset{\imath}{\hookrightarrow}\Lambda(\Z_p)\simeq\Dist(\Z_p,\Z_p)=\Meas(\Z_p,\Z_p)\hookrightarrow\Hom_{\text{cont}}(C(\Gfrak,\C_p),\C_p).\]

\section{$p$-adic measure attached to compatible systems of local units}
As we know, the $p$-adic zeta function is associated with the cyclotomic units. Those units come in compatible systems, i.e. they are elements of
\[\curlU_\infty=\varprojlim \curlU_n,\]
where $\curlU_n$ is the group of local units in $\curlK_n=\Q_p(\mu_{p^{n+1}})$. The first step in building this pseudo-measure is to pass from units to power series via the Coleman power series. Then one uses the map
\[\curlL:W\rarr R^{\psi=0}\]
to get a power series which corresponds, under the inverse Mahler transform, to a measure on $\curlG$. As we will see, applying the map $\curlL$ is essentially like taking the $\log$ and then restricting it.

\subsection{The map $\tilde{\curlL}:\curlU_\infty\rarr\Lambda(\curlG)$}
The main technical tool to pass from units to $p$-adic measures is the Coleman power series attached to compatible systems of units.

\begin{theorem}
	For each $u=(u_n)\in\curlU_\infty$, there exists a unique power series $f_u(T)\in R$ such that $f_u(\pi_n)=u_n$, where
	\[\pi_n=\zeta_n-1\]
	for some generator $(\zeta_n)$ of the Tate module $T_p(\mu)$.
\end{theorem}
\begin{proof}
	This is Theorem $2.1.2$ in \cite{CS}, which is proved in Chapter $2$.
\end{proof}

Recall that one can define a norm operator
$\curlN:R\rarr R$
as
\[(\varphi\circ\curlN)(g)(T)=\prod_{\xi\in\mu_p}g(\xi(1+T)-1).\]
The image of $\curlU_\infty$ under the map $u\mapsto f_u$ of the Theorem is
\[W=\{g\in R^\times|\curlN(g)=g\}.\]
See \cite[Corollary 2.3.7]{CS}. This gives an isomorphism
\[\xymatrix{
\curlU_\infty\ar[d]_{\text{C.P.S.}}^{\wr}\\
W
}\]
which also respects the action of $\curlG$ on both sides (recall that $g\in\curlG$ acts on $R$ by sending $T$ to $(1+T)^{\chi(g)}-1$).

The map
\[\curlL:W\rarr R^{\psi=0}\]
is defined as
\[\curlL(g)(T)=\frac{1}{p}\log\left (\frac{g(T)^p}{\varphi(g)(T)}\right )\]
in Lemma $2.5.1$.\footnote{Actually, the map is defined on $R^\times$, not just $W$, but the image of $\curlL$ lies in $R^{\psi=0}$.} One can this of this map as the restriction of the logarithm of power series in $W$. Indeed, we \emph{formally} have
\begin{align*}
	\curlL(g)(T)	&= \frac{1}{p}\log\left (\frac{g(T)^p}{\varphi(g)(T)}\right ) 	&\text{(by definition)}\\
					&= \log g(T) - \frac{1}{p}\log\varphi(g)(T)						&\text{(formally)}\\
					&= \log g(T) - \frac{1}{p}\log(\varphi\circ\curlN)(g)(T)		&\text{(since $g\in W$)}\\
					&= \log g(T) - \frac{1}{p}\log\prod_{\xi\in\mu_p}g(\xi(1+T)-1)	&\text{(by definition of $\varphi\circ\curlN$)}\\
					&= \log g(T) - \frac{1}{p}\sum_{\xi\in\mu_p}\log g(\xi(1+T)-1)	&\text{(formally)}\\
					&= \log g(T) - (\varphi\circ\psi)(\log g(T))					&\text{(by definition of $\varphi\circ\psi$)}\\
					&= (\Res\log)(g)(T))											&\text{(by definition of $\Res$)}
\end{align*}
so that we could define a map
\[\Res\log\eqdef\curlL.\]
Note that this is just notation, since the logarithm map is not necessarily well-defined on all $W$.

At this point, we have the following diagram of maps
\[\xymatrix{
	\curlU_\infty\ar[d]_{\text{C.P.S.}}^{\wr}	&\\
	W\ar[r]^{\Res\log}							& R^{\psi=0}
}\]
Using the isomorphism $\Lambda(\curlG)\simeq R^{\psi=0}$ of the previous section, we can lift the map $\curlL$ to a map $\tilde{\curlL}:\curlU_\infty\rarr \Lambda(\curlG)$, which we denote $\widetilde{\Res\log}$:
\[\xymatrix{
	\curlU_\infty\ar[d]_{\text{C.P.S.}}^{\wr}\ar[r]^{\widetilde{\Res\log}}	&\Lambda(\curlG)\ar[d]_{\wr}^{\tilde{\curlM}}\\
	W\ar[r]^{\Res\log}							& R^{\psi=0}\\
}\]

\subsection{Moments of $p$-adic measures obtained via $\tilde{\curlL}$}
The moments of the $p$-adic measures obtained via $\tilde{\curlL}$ are related to the so-called higher logarithm derivative map. More precisely, we prove Proposition $3.5.2$ of \cite{CS} in this section, i.e. that
\[\int_{\curlG}\chi(g)^k\d\tilde{\curlL}(u)=\delta_k(u),\]
where
\[\delta_k(u)=\left (D^{k-1}\left ((1+T)\frac{f'_u(T)}{f_u(T)}\right )\right )_{T=0}.\]
The map $\delta_k(u)$ is called the higher logarithmic derivative map.

First, recall that the map
\[\Delta(g)(T)=(1+T)\frac{g'(T)}{g(T)}\]
takes $W$ to $R^{\psi=1}=\{g\in R|\psi(g)=g\}$ (this is Lemma $2.4.5$ in \cite{CS}). Applying the operator $1-\varphi$ (denoted $\theta$ in \cite{CS}), which is just the restriction operator, since
\[1-\varphi=1-\varphi\circ\psi=\Res\]
on $R^{\psi=1}$, we fall in $\R^{\psi=0}$. In the proof Theorem $2.5.2$ of \cite{CS}, we learn that the diagram
\[\xymatrix{
	W\ar[d]_{\Delta}\ar[r]^{\curlL}	&R^{\psi=0}\ar[d]^{D}\\
	R^{\psi=1}\ar[r]^{\theta}		&R^{\psi=0}
}.\]
Using our notation, this is just saying that the $D$ and $\Res$ operators commute:
\[D\circ\Res\log=\Res\circ D\log.\]

Altogether, we have the following commutative diagram
\[\xymatrix{
	\curlU_\infty\ar[d]_{\text{C.P.S.}}^{\wr}\ar[r]^{\widetilde{\Res\log}}	&\Lambda(\curlG)\ar[d]_{\wr}^{\tilde{\curlM}}\\
	W\ar[d]_{D\log}\ar[r]^{\Res\log}	&R^{\psi=0}\ar[d]^{D}\\
	R^{\psi=1}\ar[r]^{\Res}		&R^{\psi=0}
}.\]

Using this diagram, we now compute the moments:
\begin{align*}
	\int_{\curlG}\chi(g)^k\d\tilde{\curlL}(u) 	&= \int_{\curlG}\chi(g)^k\d\widetilde{\Res\log}(u) 	&\text{(notation)}\\
												&= \int_{\Z_p}x^k\d\curlY(\Res\log f_u)				&\text{(commutativity of top square)}\\
												&= \int_{\Z_p}x^{k-1}\d x\curlY(\Res\log f_u)		&\\
												&= \int_{\Z_p}x^{k-1}\d \curlY(D\circ\Res\log f_u)	&\text{(formula \ref{eq:momentsandD})}\\
												&= \int_{\Z_p}x^{k-1}\d \curlY(\Res\circ D\log f_u)	&\text{(commutativity of bottom square)}\\
												&= \int_{\Z_p}x^{k-1}\d \Res\curlY(D\log f_u)		&\\
												&= (1-p^{k-1})\int_{\Z_p}x^{k-1}\d \curlY(D\log f_u)&\text{(formula \ref{eq:momentsofres}, since $D\log f_u\in R^{\psi=1}$)}\\
												&= (1-p^{k-1})(D^{k-1}(D\log f_u))_{T=0}			&\text{(formula \ref{eq:momentsandD})}\\
												&= (1-p^{k-1})\delta_k(u)							&\text{(by definition of $\delta_k(u)$)}\\
\end{align*}


\subsection{Measures attached to generators of the Tate module $T_p(\mu)$}
As one can see, any generator $(\zeta_n)$ of the Tate module $T_p(\mu)$ is a norm compatible sequence of units, hence can be viewed as an element of $\curlU_\infty$. Can we obtain interesting measure from those elements? Unfortunately, no. To see this, note that since by definition $\pi_n=\zeta_n-1$, it is clear that the Coleman power series of $(\zeta_n)\in\curlU_\infty$ is simply $1+T$. But then
\[\tilde{\curlL((\zeta_n))}=\curlL(1+T)=\frac{1}{p}\log\left (\frac{(1+T)^p}{(1+(1+T)^p-1)}\right )=\frac{1}{p}\log 1=0,\]
so the corresponding measure on $\curlG$ is the zero measure! In fact, one has the following \emph{Fundamental Exact Sequence} of $\curlG$-modules
\[\xymatrix{
	0\ar[r]	&\mu_{p-1}\times T_p(\mu)\ar[r]	&\curlU_\infty\ar[r]^{\tilde{\curlL}}	&\Lambda(\curlG)\ar[r]^\beta	&T_p(\mu)\ar[r]	&0
},\]
where the map $\beta$ sends $\lambda$ to $(\zeta_n)^{\int_\curlG\chi\d\lambda}$ (see \cite[Theorem 3.5.1]{CS}). Note this sequence is the main ingredient in the proof of Iwasawa's theorem (Theorem $4.4.1$ in \cite{CS}).

The proof this sequence is exact follows essentially from the exactness of the sequence
\[\xymatrix{
	0\ar[r]	&\Z_p\ar[r]	&R^{\psi=1}\ar[r]^{\theta=\Res}	&R^{\psi=0}\ar[r]^{\text{ev}_{T=0}}	&\Z_p\ar[r]	&0,
}\]
which is proved in \cite[Lemma 2.4.3]{CS}. Indeed, the exactness of the sequence
\begin{equation}\label{seq:Dlog}
	\xymatrix{0\ar[r]	&\mu_{p-1}\ar[r]	&W\ar[r]^{D\log}	& R^{\psi=1}\ar[r]	&0}
\end{equation}
allows us to lift this sequence to obtain the following commutative diagram
\[\xymatrix{
	0\ar[r]	&A\ar[r]	&W\ar[r]^{\Res\log}\ar[d]_{D\log}&R^{\psi=0}\ar[r]^{\alpha}\ar[d]^{D}		&\Z_p\ar[r]	&0\\
	0\ar[r]	&\Z_p\ar[r]	&R^{\psi=1}\ar[r]^{\Res}	&R^{\psi=0}\ar[r]^{\text{ev}_{T=0}}	&\Z_p\ar[r]	&0,
}\]
where
\[A=\{\xi(1+T)^a|\xi\in\mu_{p-1},a\in\Z_p\}\]
and $\alpha(g)=(Dg)(0)$ (see \cite[Theorem 2.5.2]{CS}). The bottom row can be simply seen as the additive version of the top row. Finally, using the isomorphisms
\[\xymatrix{
	\curlU_\infty\ar[d]_{\text{C.P.S.}}^{\wr}	&\Lambda(\curlG)\ar[d]_{\wr}^{\tilde{\curlM}}\\
	W	&R^{\psi=0}\\
}\]
one can further lift this sequence to obtain
\[\xymatrix{
	0\ar[r]	&\mu_{p-1}\times T_p(\mu)\ar[r]	&\curlU_\infty\ar[r]^{\tilde{\curlL}}\ar[d]_{\text{C.P.S.}}^{\wr}	&\Lambda(\curlG)\ar[r]^\beta\ar[d]_{\wr}^{\tilde{\curlM}}	&T_p(\mu)\ar[r]	&0\\
	0\ar[r]	&A\ar[r]	&W\ar[r]^{\Res\log}\ar[d]_{D\log}&R^{\psi=0}\ar[r]^{\alpha}\ar[d]^{D}		&\Z_p\ar[r]	&0\\
	0\ar[r]	&\Z_p\ar[r]	&R^{\psi=1}\ar[r]^{\Res}	&R^{\psi=0}\ar[r]^{\text{ev}_{T=0}}	&\Z_p\ar[r]	&0.
}\]

In brief, one needs to work a little bit harder to obtain non-trivial measures on $\curlG$. This is where the cyclotomic units come in!


\begin{thebibliography}{DDDD}
\bibitem[CS]{CS}
    {\scshape\itshape Coates, J., Sujatha, R.}, \emph{Cyclotomic Fields and zeta Values}, Springer Monographs in Mathematics, Springer, 2006.

\end{thebibliography}
\end{document}

