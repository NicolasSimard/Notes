\documentclass{beamer}
\usetheme{Singapore}

% --------------------------------------------------------------------
% Packages pour la couleur
\usepackage{color}

% --------------------------------------------------------------------
% Voici certains packages souvent utilisés.
\usepackage{graphicx}		% Importation du package permettant d'inclure d'images dans le document.
\usepackage{amsmath, amsfonts}	% Pour écrire selon les standards de l'AMS.
\usepackage{epstopdf}		% Package permettant l'inclusion d'images eps, pour la compilation en pdf.
\usepackage{epsfig}
\usepackage{hyperref}
%\usepackage{subfig}
\usepackage{multirow}

% --------------------------------------------------------------------
% Le package "palatino" charge la police Palatino en mode texte et le package "euler" charge la police Euler en mode mathématique. Pour retrouver les polices par défaut, effacez les deux lignes de commandes qui suivent.
%\usepackage{palatino}
%\usepackage{euler}

% --------------------------------------------------------------------
% --------------------------------------------------------------------
% Commande rapide
\newcommand{\C}{\mathbb{C}}
\newcommand{\F}{\mathbb{F}}
\newcommand{\R}{\mathbb{R}}
\newcommand{\Q}{\mathbb{Q}}
\newcommand{\N}{\mathbb{N}}
\newcommand{\Z}{\mathbb{Z}}
\newcommand{\cH}{\mathcal{H}}
\newcommand{\cO}{\mathcal{O}}
\newcommand{\cA}{\mathcal{A}}

\newcommand{\abcdmat}{\begin{pmatrix}
a & b \\ 
c & d
\end{pmatrix}}
\newcommand{\fa}{\mathfrak{a}}
\newcommand{\fb}{\mathfrak{b}}
\newcommand{\fc}{\mathfrak{c}}
\newcommand{\fp}{\mathfrak{p}}
\newcommand{\tpsi}{\theta_\psi}
\newcommand{\ClK}{\text{Cl}_K}

\DeclareMathOperator{\Ind}{Ind}
\DeclareMathOperator{\Gal}{Gal}

\newtheorem{thm}{Theorem}
\newtheorem{defn}[thm]{Definition}
\newtheorem{prop}[thm]{Proposition}
\newtheorem{coro}[thm]{Corollary}
\newtheorem{lem}[thm]{Lemma}
\newtheorem{conj}[thm]{Conjecture}
\newtheorem{rem}[thm]{Remark}
\newtheorem{ex}[thm]{Example}
\newtheorem{qu}{Question}

\DeclareMathOperator{\reg}{reg}

\title{Petersson Inner Product of Theta Series}
\subtitle{An experimental approach}
\author{Nicolas \textsc{Simard}}
\institute{McGill University}
\date{December 1st, 2017}

\begin{document}

\frame{
	\titlepage
}

\section{Introduction}
\subsection{Motivation}
\frame{
	\frametitle{Stark's observation}
	Let $K=\Q(\sqrt{-23})$ and let $H$ be the HCF of $K$.
	Let
	\[\psi:\Gal(H/K)\rightarrow \text{GL}_1(\C)\]
	be a non-trivial one-dimensional Artin representation and let
	\[\rho =\Ind_{K}^{\Q}\psi:\Gal(H/\Q)\rightarrow\text{GL}_2(\C)\]
	be the induced representation. 
}

\frame{
	\frametitle{Stark's observation (cont.)}
	By Deligne-Serre, one has
	\[L(\rho, s) = L(\theta_\psi,s),\]
	where
	\[\tpsi(q) = \eta(q)\eta(23q) = q\prod_{n\geq 1}(1-q^n)(1-q^{23n})\in M_1(\Gamma_0(23),\chi_{-23}).\]
	Then Stark proves that
	\[\langle\tpsi,\tpsi\rangle = 3\log\varepsilon,\]
	where $\varepsilon$ is the real root of
	\[x^3-x-1.\]
}

\subsection{Structure of the Talk}
\frame{
	\frametitle{Structure of the talk}
	\tableofcontents
}

\section[Weight One]{Petersson Inner Product of Weight One Theta Series}
\subsection{Explicit Formulas}

\frame{
	\frametitle{Notation}
	Throughout this talks, let
	\begin{itemize}
		\item $K$ be an imaginary quadratic field of discriminant $D$,
		\item $H$ be the Hilbert class field of $K$,
		\item $h_K$ be the class number of $K$,
		\item $w_K$ be the number of roots of unity in $K$ and
		\item $ClK$ be the class group of $K$.
	\end{itemize}
}

\frame{
	\frametitle{Weight one theta series}
	Let $\psi$ be a class character of $K$, i.e. a homomorphism
	\[\psi:\ClK\rightarrow\C^\times.\]
	Then
	\[\theta_\psi(q) = \sum_{\fa}\psi(\fa)q^{N(\fa)}\in M_1(\Gamma_0(|D|),\chi_D).\]
	Moreover, $\theta_\psi$ is an eigenform for all Hecke operators.
	
	If $\psi^2 = 1$, $\theta_\psi$ is an Eisenstein series.	
	
	If $\psi^2\neq 1$, $\theta_\psi$ is a cusp form (in fact, a newform).	
}

\frame{
	\frametitle{Stark's example}
	Let
	\[K=\Q(\sqrt{-23})\]
	and let $\psi$ be a non-trivial class character as above.
	Then
	\[\text{Stark's }\tpsi =\text{our }\tpsi\in M_1(\Gamma_0(23),\chi_{-23}).\]
	Note that if $\psi'$ is the other non-trivial class character, then
	\[\theta_\psi = \theta_{\psi'}.\]
}

\frame{
	\frametitle{Petersson inner product of weight one theta series}
	The Petersson inner product of any two $f,g\in S_k(\Gamma_0(N),\chi)$ is defined as
	\[\langle f,g\rangle =\iint_{\Gamma_0(N)\setminus\cH}f(x+iy)\overline{g(x+iy)}y^k\text{d}\mu.\]
	Then
	\begin{prop}[S.]
		Let $\psi$ be a class character which is not a genus character. Then
		\[\langle\theta_\psi,\theta_\psi\rangle = \frac{-h_K}{3w_K^2}\sum_{\cA\in\ClK}\psi^2(\cA)\log N(\cA)^6|\Delta(\cA)|.\]
	\end{prop}
}


\subsection{Generalizing Stark's Observation}
\frame{
	\frametitle{Siegel units}
	Let $\fa$ be a fractional ideal of $K$ and define
	\[|\delta_\cA|=(N(\fa)^6|\Delta(\cO_K)/\Delta(\fa^{-1})|)^{h_K},\]
	where $\fa$ is any ideal in the class $\cA$.
	Then $|\delta_\cA|$ is a unit in $H$.
	
	Since $\psi^2$ is not trivial, one sees that
	\[\langle \theta_\psi,\theta_\psi\rangle = \frac{1}{3w_K^2}\sum_{\cA\in\ClK}\psi^2(\cA)\log |\delta_\cA|,\]
	where $\{\fa_1,\dots,\fa_{h_K}\}$ is a set of class representatives for $\ClK$.
}

\frame{
	\frametitle{What about Stark's observation?}
	One can write
	\[\langle \theta_\psi,\theta_\psi\rangle=h_K\log\kappa_\psi,\]
	where
	\[\kappa_\psi = \prod_{j=1}^{h_K}\Phi(\fa_j)^{-\psi^2(\fa_j)}\]
	with
	\[\Phi(\fa) = \sqrt{N(\fa)}|\Delta(\fa)|^{1/12}.\]
	\begin{qu}
		Is $\kappa_\psi$ a unit in $H$?
	\end{qu}
	Calcs in class nbr 3, 4, 5, 6.
}

\frame{
	\frametitle{Generalizing Stark's Observation}
	\begin{prop}[S.]
	Let $\psi$ be a class character such that $\psi^2$ is a non-trivial character with rational real part.
	Then $\kappa_\psi$ is an algebraic integer which is a unit.
	Moreover, if $\psi^2$ is a non-trivial genus character corresponding to the factorisation $D = D_1D_2$, with $D_1>0$ say, then
	\[\kappa_\psi = \epsilon_{D_1}^{\frac{4h_{D_1}h_{D_2}}{w_Kw_{D_2}}},\]
	where $\epsilon_{D_1}$ is the fundamental unit of $\Q(\sqrt{D_1})$, $h_{D_j}$ is the class number of $\Q(\sqrt{D_j})$ and $w_{D_2}$ is the number of roots of unity in $\Q(\sqrt{D_2})$.
\end{prop}
}

\frame{
	\frametitle{Examples}
	\only<1->{If $K = \Q(\sqrt{-23})$, the Proposition implies that $\kappa_\psi$ is a unit.
	But is it in the Hilbert class field?
	}
	
	\only<2->{If $K = \Q(\sqrt{-39})$, the Proposition implies
	\[\kappa_\psi = \epsilon_{13}^\frac{1}{3},\]
	which is \emph{not} in the Hilbert class field.
	}
}
\frame{
	\frametitle{Stark's observation: the final word?}
	Note that $\psi^2$ has rational real part if and only if its order divides $4$ or $3$.
	\begin{coro}
		Suppose that $K$ has class number divisible by $2$ or $3$.
		Then there exists a class character $\psi$ such that
		\[\kappa_\psi\]
		is a unit.
	\end{coro}
	
	\begin{qu}
		Is the converse true?
	\end{qu}
}


\section[Higher Weight]{Petersson Inner Product of Higher Weight Theta Series}

\section{Spaces of Theta Series}

\frame{
	\frametitle{Thank you!}
	
	Presentation and notes available at: \url{https://github.com/NicolasSimard/Notes}
	
	Code available at : \url{https://github.com/NicolasSimard/ENT}
	
	Or from my webpage: \url{http://www.math.mcgill.ca/nsimard/}
}
\end{document}
