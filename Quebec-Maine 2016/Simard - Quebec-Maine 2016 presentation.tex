\documentclass{beamer}
\usetheme{Singapore}

% --------------------------------------------------------------------
% Packages pour la couleur
\usepackage{color}

% --------------------------------------------------------------------
% Packages pour les accents français.
\usepackage[latin1]{inputenc}	% L'option "latin1" désigne l'encodage ISO-8859-1, typique sous Linux.
\usepackage[T1]{fontenc}	% Pour les accents
\usepackage[frenchb]{babel}

% --------------------------------------------------------------------
% Voici certains packages souvent utilisés.
\usepackage{graphicx}		% Importation du package permettant d'inclure d'images dans le document.
\usepackage{amsmath, amsfonts}	% Pour écrire selon les standards de l'AMS.
\usepackage{epstopdf}		% Package permettant l'inclusion d'images eps, pour la compilation en pdf.
\usepackage{epsfig}
\usepackage{hyperref}
%\usepackage{subfig}
\usepackage{multirow}

% --------------------------------------------------------------------
% Le package "palatino" charge la police Palatino en mode texte et le package "euler" charge la police Euler en mode mathématique. Pour retrouver les polices par défaut, effacez les deux lignes de commandes qui suivent.
\usepackage{palatino}
\usepackage{euler}

% --------------------------------------------------------------------
% --------------------------------------------------------------------
% Commande rapide
\newcommand{\C}{\mathbb{C}}
\newcommand{\F}{\mathbb{F}}
\newcommand{\R}{\mathbb{R}}
\newcommand{\Q}{\mathbb{Q}}
\newcommand{\N}{\mathbb{N}}
\newcommand{\Z}{\mathbb{Z}}
\renewcommand{\H}{\mathcal{H}}
\renewcommand{\O}{\mathcal{O}}

\newcommand{\abcdmat}{\begin{pmatrix}
a & b \\ 
c & d
\end{pmatrix}}
\newcommand{\ida}{\mathfrak{a}}
\newcommand{\idb}{\mathfrak{b}}
\newcommand{\idc}{\mathfrak{c}}
\newcommand{\p}{\mathfrak{p}}
\newcommand{\tha}{\theta_\mathfrak{a}}
\newcommand{\thb}{\theta_\mathfrak{b}}
\newcommand{\tpsi}{\theta_\psi}
\newcommand{\ClK}{\text{Cl}_K}

\newtheorem{thm}{Theorem}
\newtheorem{defn}[thm]{Definition}
\newtheorem{prop}[thm]{Proposition}
\newtheorem{coro}[thm]{Corollary}
\newtheorem{lem}[thm]{Lemma}
\newtheorem{conj}[thm]{Conjecture}
\newtheorem{rem}[thm]{Remark}
\newtheorem{ex}[thm]{Example}
\newtheorem{qu}{Question}

\DeclareMathOperator{\reg}{reg}

\title{Petersson Inner Product of Binary Theta Series}
\subtitle{A computational approach}
\author{Nicolas \textsc{Simard}}
\institute{McGill University}
\date{October 8th, 2016}


\begin{document}

\frame{
	\titlepage
}

\section{Introduction and Background}
\subsection{Introduction}
\frame{
	\frametitle{Motivation: Stark's remark}
	In \emph{"L-functions at $s=1$. II. Artin L-functions with Rational Characters"}, Stark makes the following remark:
	\begin{quotation}
		An application of Theorem $1$ gives
		\[L'(0,\chi,H/\Q) = \log\epsilon,\]
		where $\epsilon$ is the real root of
		\[x^3-x-1=0\]
		Actually, it is easier to note that $L(1,\chi,H/\Q)$ is the residue at $s=1$ of the zeta function of the real quadratic subfield of $H$. In any case,
		\[\langle f,f\rangle = 3\log\epsilon.\]
	\end{quotation}
}

\subsection{Basic definitions}
\frame{
	\frametitle{Eisenstein series}
	Let $k\geq2$ be an even integer. Define
	\[G_k(z) = -\frac{B_k}{2k}+\sum_{n=1}^\infty\sigma_{k-1}(n)q^n,\]
	for $k\geq 4$ and
	\[G_2(z) = \frac{1}{8\pi\Im(z)}-\frac{1}{24}+\sum_{n=1}^\infty\sigma(n)q^n,\]
	where
	\[\sigma_{k-1}(n)=\sum_{d|n}d^{k-1}.\]
}

\frame{
	\frametitle{Petersson inner product}
	Let $f,g\in S_k(\Gamma_0(N),\chi)$ be two cusp forms. The Petersson inner product of $f$ and $g$ is defined as
	\[\langle f,g\rangle =\int\int_{\Gamma_0(N)\setminus\H}f(x+iy)\overline{g(x+iy)}y^k\text{d}\mu,\]
	where
	\[\text{d}\mu=\frac{\text{d}x\text{d}y}{y^2}\]
	is the $\text{SL}_2(\R)$-invariant measure on $\H$.
	Note that the integral does not converge if both $f$ and $g$ are not cusp forms.
}

\subsection{Theta Series}
\frame{
	\frametitle{Theta series attached to ideals}
	Let $K$ be an imaginary quadratic field of discriminant $D<-4$ and let $\O_K$ be its ring of integers. Fix an integer $\ell\geq0$.
	
	To each integral ideal $\ida$ of $K$, one can attach the following theta series:
	\[\tha^{(2\ell)}(z)=\sum_{x\in\ida}x^{2\ell}q^{N(x)/N(\ida)}\in M_{2\ell+1}(\Gamma_0(|D|),\chi_D),\]
	where $\chi_D$ is the Kronecker symbol. If $\ell>0$, then
	\[\tha\in S_{2\ell+1}(\Gamma_0(|D|),\chi_D).\]
}

\frame{
	\frametitle{Theta series attached to Hecke characters of $K$}
	Let $I_K$ denote the group of fractional ideals of $K$ and let $\psi$ be a Hecke character of infinity type $2\ell$, i.e. a homomorphism
	\[\psi:I_K\longrightarrow \C^\times\]
	such that
	\[\psi((\alpha))=\alpha^{2\ell},\hspace{1cm}\forall\alpha\in K^\times.\]
	Then one defines
	\[\tpsi=\sum_{\ida\subseteq\O_K}\psi(\ida)q^{N(\ida)}\in M_{2\ell+1}(\Gamma_0(|D|),\chi_D),\]
	where $\chi_D$ is the Kronecker symbol. If $\psi^2\neq 1$, then
	\[\tpsi\in S_{2\ell+1}(\Gamma_0(|D|),\chi_D).\]
}

\frame{
	\frametitle{Stark's example}
	Let
	\[K=\Q(\sqrt{-23})\]
	and let $\psi$ be a non-trivial Hecke character of infinity type $0$, i.e. a non-trivial character of the class group. Then
	\[\text{Stark's }f =\text{our }\tpsi\in M_1(\Gamma_0(23),\chi_{-23}).\]
}

\subsection{Some questions}
\frame{
	\frametitle{Some questions}
	Keeping Stark's example in mind, we have the following questions:
	\begin{itemize}
		\item Can we find explicit formulas for the Petersson inner product of those theta series (whenever it makes sense)?
		\item Can we efficiently compute it?
		\item Can we use those formulas/computations to study the arithmetic properties of those quantities?
	\end{itemize}
	\only<2->{
	The main question is
	\begin{qu}
		Can we $p$-adically interpolate those formulas for $\ell>0$ and take the limit as $\ell\rightarrow 0$ $p$-adically to obtain the weight one case?
	\end{qu}
	}
}

\section{Formulas}
\subsection{The case $\ell=0$}
\frame{
	\frametitle{The case $\ell=0$}
	\begin{theorem}
		Let $\psi$ be a Hecke character of infinity type $0$ which is not a genus character. Then
		\begin{align*}
		\langle\tpsi,\tpsi\rangle 	&= -h_K\sum_{[\ida]\in\ClK}\psi^2(\ida)\log(N(\ida)^{1/2}|\eta(\ida)|^2)\\
									&= h_K\log\prod_{[\ida]\in\ClK}(N(\ida)^{1/2}|\eta(\ida)|^2)^{-\psi(\ida)^2}.
		\end{align*}
	\end{theorem}
	Here,
		\[\eta(z)=exp(2\pi i/24)\prod_{n=1}^\infty(1-q^n).\]
	
}

\subsection{The case $\ell>0$}
\frame{
	\frametitle{Petersson norm of the $\tpsi$ (with $\ell>0$)}
	\begin{theorem}
		Let $\psi$ be a Hecke character of $K$ of infinity type $2\ell$, where $\ell>0$. Then
		\[\langle\tpsi,\tpsi\rangle = h_K(|D|/4)^\ell\sum_{[\ida]\in\ClK}\psi^2(\ida)\partial^{2\ell-1}G_2(\ida).\]
	\end{theorem}
	Here,
	\[\partial f=\frac{1}{2\pi i}\frac{\partial f}{\partial z}-\frac{k}{4\pi\Im(z)}f\]
	is the Shimura-Maass differential operator, which preserves the graded algebra of almost holomorphic modular forms.
}

\frame{
	\frametitle{Petersson inner product of the theta series $\tha$}
	\begin{coro}
		Let $\ida$ and $\idb$ be ideals of $K$ and suppose $\ell>0$. Then
		\[\langle\tha,\thb\rangle=C_K^{(2\ell)}N(\idb)^{2\ell}\sum_{\ida\idb^{-1}\idc^2=\lambda_\idc\O_K}\lambda_\idc^{2\ell}\partial^{2\ell-1}G_2(\idc),\]
		where
		\[C_K^{(2\ell)}=4(|D|/4)^\ell.\]
	\end{coro}
}

\subsection{The case $\ell=0$ revisited}
\frame{
	\frametitle{Formally obtaining the case $\ell=0$ from the case $\ell>0$}
	Strictly speaking, the formula
	\[\langle\tpsi,\tpsi\rangle = h_K(|D|/4)^\ell\sum_{[\ida]\in\ClK}\psi^2(\ida)\partial^{2\ell-1}G_2(\ida).\]	
	does not make sense for $\ell=0$, since the expression
	\[\partial^{-1}G_2\]
	is not well-defined. \only<2->{However, we observe that
	\[\partial_0 \log(\Im(z)^{1/2}|\eta(z)|^2)=-G_2(z),\]
	so
	\["\partial^{-1}G_2(z)=-\log(\Im(z)^{1/2}|\eta(z)|^2)"\]
	and we \emph{formally} obtain the case $\ell=0$ from the case $\ell>0$.}
}

\section{Examples}
\subsection{The case $\ell=0$}
\frame{
	\frametitle{$K=\Q(\sqrt{-23})$ (class number $3$, one genus): $\ell=0$}
	For $\psi$ a non-trivial Hecke character of infinity type $0$, the explicit formula in case $\ell=0$ gives
	\[\langle f,f\rangle=\langle\tpsi,\tpsi\rangle=3\log \epsilon,\]
	where
	\[\epsilon=\prod_{[\ida]\in\ClK}(N(\ida)^{1/2}|\eta(\ida)|^2)^{-\psi(\ida)^2}\]
	is the real root of
	\[x^3-x-1\]
	and generates the Hilbert class field of $K$.
}

%\frame{
%	\frametitle{$K=\Q(\sqrt{-47})$ (class number $5$, one genus): $\ell=0$}
%	For this field and any non-trivial class character $\psi$,
%	\[\prod_{[\ida]\in\ClK}(N(\ida)^{1/2}|\eta(\ida)|^2)^{-\psi(\ida)^2}\]
%	does \emph{not} seem to be a unit...
%}

\frame{
	\frametitle{Class field theory}
	\begin{theorem}
	Let $D$ be a prime discriminant and let $H$ be the Hilbert class field of $K=\Q(\sqrt{D})$. Then
	\[\prod_{\psi\neq 1}\langle\tpsi,\tpsi\rangle = \frac{2}{w_H}h_K^{h_k-2}h_H\reg H,\]
	where $w_H=\vert\mathcal{O}_H^\times\vert$, $h_H$ is the class number of $H$ and $\reg H$ is the regulator of $H$.	
	\end{theorem}
}

\subsection{The case $\ell>0$}
\frame{
	\frametitle{Algebraic part of the Petersson inner product for $\ell>0$}
	Let
	\[\Omega_K=\frac{1}{\sqrt{4\pi|D|}}\left(\prod_{j=1}^{|D|-1}\Gamma\left(\frac{j}{|D|}\right)^{\chi_D(j)}\right)^{w_K/4h_k}\]
	be the Chowla-Selberg period attached to $K$.
	\begin{coro}
		For $\ell>0$, the complex numbers
		\[\frac{\langle\tpsi,\tpsi\rangle}{\Omega_K^{4\ell}}\hspace{0.5cm}\text{and}\hspace{0.5cm}\frac{\langle\tha,\thb\rangle}{\Omega_K^{4\ell}}\]
		are algebraic.
	\end{coro}
}

\frame{
	\frametitle{$K=\Q(\sqrt{-23})$ (class number $3$, one genus): $\ell>0$}
	In $K$, the prime $2$ splits as
	\[2\O_K=\p_2\bar{\p}_2\]
	and
	\[\ClK=\lbrace 1,[\p_2],[\bar{\p}_2]\rbrace.\]
	Moreover, we have $\langle\theta_{\bar{\p}_2},\theta_{\O_K}\rangle=\overline{\langle\theta_{\p_2},\theta_{\O_K}\rangle}$. We will focus on
	\[\langle\theta_{\O_K},\theta_{\O_K}\rangle/\Omega_K^{4\ell}.\]
}

\frame{
	\frametitle{$K=\Q(\sqrt{-23})$ (class number $3$, one genus): $\ell>0$}
	Consider the algebraic number
	\[a(\ell)=\langle\theta_{\O_K},\theta_{\O_K}\rangle/\Omega_K^{4\ell}.\]
	For $\ell=1,2,4$ and $5$, we find that $a(\ell)^3$ is a root of a monic cubic polynomial and generates the Hilbert class field over $K$. 
	\begin{ex}
	$a(1)$ is a root of the polynomial
	\[x^9 - 2816x^6 - 905216x^3 - 89915392.\]
	\end{ex}
}

\frame{
	\frametitle{$K=\Q(\sqrt{-23})$ (class number $3$, one genus): $\ell>0$}
	Consider the algebraic number
	\[a(\ell)=\langle\theta_{\O_K},\theta_{\O_K}\rangle/\Omega_K^{4\ell}.\]
	For $\ell=3,6$ and $9$, we find that $a(\ell)$ is a root of a cubic polynomial and generates the Hilbert class field over $K$.
	\begin{ex}
	$a(3)$ is a root of
	\[x^3 - 6740x^2 - 169034720x - 1027491892288.\]
	\end{ex}
}

\frame{
	\frametitle{$K=\Q(\sqrt{-23})$ (class number $3$, one genus): $\ell>0$}
	A few computations of the Gramm matrix for this basis.
	\begin{tabular}{|c|c|}
	\hline
	$\ell$ & $\det(\langle\theta_{\ida_i}^{(2\ell)},\theta_{\ida_j}^{(2\ell)}\rangle)_{\ida_i,\ida_j\in\ClK}/(\Omega_K^{4\ell})^3$ \\
	\hline
	$1$ & $-2^{10}23$ \\
	\hline
	$2$ & $-2^{14}19\cdot23\cdot619$ \\ 
	\hline
	$3$ & $-2^{18}5^{2}11\cdot23\cdot337\cdot27299$ \\ 
	\hline
	$4$ & $-2^{22}7^{2}23\cdot163\cdot2113\cdot117741979$ \\ 
	\hline
	$5$ & $-2^{26}5^{3}23\cdot229\cdot23761\cdot808991\cdot20338663$ \\ 
	\hline
	$6$ & $-2^{30}5^{2}11^{2}13\cdot19\cdot23\cdot67^{2}101\cdot868697\cdot505912247899$ \\ 
	\hline
	\end{tabular}
}

\frame{
	\frametitle{$K=\Q(\sqrt{-23})$ (class number $3$, one genus): $\ell>0$}
	Consider now the algebraic number
	\[N(\psi,\ell) = \langle\theta_{\psi},\theta_{\psi}\rangle/\Omega_K^{4\ell}\]
	For $\ell=1,2,4$ and $5$, the numbers $N(\psi_i,\ell)$, for $0\leq i\leq 2$, are distinct and their cube are the three real roots of a monic cubic polynomial. 
	\begin{ex}
		The numbers $N(\psi_i,1)^3$, for $0\leq i\leq 2$, are the three roots of the irreducible polynomial
	\[x^3 - 6966x^2 + 11569230x - 239483061.\]
	\end{ex}
}

\frame{
	\frametitle{$K=\Q(\sqrt{-23})$ (class number $3$, one genus): $\ell>0$}
	Consider now the algebraic number
	\[N(\psi,\ell) = \langle\theta_{\psi},\theta_{\psi}\rangle/\Omega_K^{4\ell}\]
	When $\ell=3,6$ and $9$, for one of the characters, say $\psi_0$, the algebraic number $N(\psi_0,\ell)$ is an \emph{integer}. For the two others, we find that their cube are the roots of a monic quadratic polynomial.
	\begin{ex}
	We have
	\[N(\psi_0,3) = 5055 = 3\cdot5\cdot337\]
	and $N(\psi_1,3)^3$ and $N(\psi_2,3)^3$ are the roots of
	\[x^2 - 16287872873193x + 30021979248651078296845875.\]
	\end{ex}
}
\frame{
	\frametitle{$K=\Q(\sqrt{-23})$ (class number $3$, one genus): $\ell>0$}
	A few computations of the Gramm matrix for this basis.
	\begin{tabular}{|c|c|}
	\hline
	$\ell$ & $\det(\langle\theta_{\psi_i},\theta_{\psi_j}\rangle)_{1\leq i,j\leq 3}/(\Omega_K^{4\ell})^3$ \\
	\hline
	$1$ & $-3^{3}23$ \\
	\hline
	$2$ & $-3^{3}19\cdot23\cdot619$ \\ 
	\hline
	$3$ & $-3^{3}5^{2}11\cdot23\cdot337\cdot27299$ \\ 
	\hline
	$4$ & $-3^{3}7^{2}23\cdot163\cdot2113\cdot117741979$ \\ 
	\hline
	$5$ & $-3^{3}5^{3}23\cdot229\cdot23761\cdot808991\cdot20338663$ \\ 
	\hline
	$6$ & $-3^{3}5^{2}11^{2}13\cdot19\cdot23\cdot67^{2}101\cdot868697\cdot505912247899$ \\ 
	\hline
	\end{tabular}
}


\section{Idea of the proof}
\subsection{Idea of the proof}
\frame{
	\frametitle{Main steps in the proof (case $\ell>0$)}
	\begin{enumerate}
		\item<1-> Use the Rankin-Selberg to prove that
		\[\langle\tpsi,\tpsi\rangle=\textcolor{gray}{\frac{4h_k}{w_k}\sqrt{|D|}\frac{\Gamma(2\ell+1)}{(4\pi)^{2\ell+1}}}L(\psi^2,2\ell+1).\]
		\item<2-> Relate Hecke L-series of imaginary quadratic fields to real-analytic Eisenstein series:
		\[L(\psi^2,2\ell+1)=\frac{1}{w_K}\sum_{[\ida]\in\ClK}\frac{\psi^2(\ida)}{N(\ida)^{4\ell-s}}G_{4\ell}(\ida,1-2\ell).\]
		\item<3-> Replace real-analytic Eisenstein series by derivatives of Eisenstein series:
		\[\partial^{2\ell-1}G_2(z)=\textcolor{gray}{(-4\pi)^{1-2\ell}\frac{\Gamma(s+2\ell+1)}{\Gamma(s+2)}}G_{4\ell}(z,1-2\ell).\]
		\item<4-> Find $\langle\tha,\thb\rangle$ using $\langle\tpsi,\tpsi\rangle$.
	\end{enumerate}
}

\section{Conclusion}
\subsection{Conclusion}
\frame{
	\frametitle{What we would like to know}
	\begin{enumerate}
		\item Can we explain what we observed in the computations?
		\item Can we say something about the Petersson inner product of non-cuspidal weight one theta series?
		
	\end{enumerate}
	\only<2->{
	But again, the main question remains
	\begin{qu}
		Can we $p$-adically interpolate the formulas for $\ell>0$ and take the limit as $\ell\rightarrow 0$ $p$-adically to obtain the weight one case?
	\end{qu}
	}
}

\frame{
	\frametitle{Thank you!}
	
	Presentation and notes available at: \url{https://github.com/NicolasSimard/Notes}
	
	Code available at : \url{https://github.com/NicolasSimard/ENT}
}
\end{document}
